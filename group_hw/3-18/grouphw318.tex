\documentclass[11pt]{exam}
\usepackage[margin=1in]{geometry}
\usepackage{amsfonts, amsmath, amssymb, amsthm}
\usepackage{mathtools}
\usepackage{enumerate}
\usepackage{listings}
\usepackage{array}

\usepackage{fullpage} % margin formatting
\usepackage{enumitem} % configure enumerate and itemize
\usepackage{amsmath, amsfonts, amssymb} % math symbols
\usepackage{xcolor, colortbl} % colors, including in tables
\usepackage{makecell} % thicker \Xhline in table
\usepackage{graphicx} % images, resizing
\usepackage[T1]{fontenc}
% sometimes needed packages
% \usepackage{hyperref} % hyperlinks
% \usepackage{logicproof} % natural deduction
\usepackage{tikz} % drawing graphs
\usepackage{algpseudocode} % pseudocode

% paragraph formatting
\setlength{\parskip}{6pt}
\setlength{\parindent}{0cm}

% newline after Solution:
\renewcommand{\solutiontitle}{\noindent\textbf{Solution:}\par\noindent}

% less space before itemize/enumerate
\setlist{topsep=0pt}

% creates \filcl to grey out cells for groupwork grading
\newcommand{\filcl}{\cellcolor{gray!25}}


% creates \probnum to get the problem number
\newcounter{probnumcount}
\setcounter{probnumcount}{1}
\newcommand{\probnum}{\arabic{probnumcount}. \addtocounter{probnumcount}{1}}

% use roman numerals by default
% \setlist[enumerate]{label={(\roman*)}}

% creates custom list environments for grading guidelines, question parts
\newlist{guidelines}{itemize}{1}
\setlist[guidelines]{label={}, left=0pt .. \parindent, nosep}
\newlist{gwguidelines}{enumerate}{1}
\setlist[gwguidelines]{label={(\roman*)}, nosep}
\newlist{qparts}{enumerate}{2}
\setlist[qparts]{label={(\alph*)}}
\newlist{qsubparts}{enumerate}{2}
\setlist[qsubparts]{label={(\roman*)}}
\newlist{stmts}{enumerate}{1}
\setlist[stmts]{label={(\roman*)}, nosep}



% in order to compile this file you need to get 'header.tex' from
% Canvas and change the line below to the appropriate file path
\input{"/Users/omerjunedi/Documents/math351/individual_hw/header.tex"}

% TODO: always change
\newcommand{\hwnum}{3-18}
\newcommand{\duedate}{March 18}
\newcommand{\qs}{\sum_{n=m}^{\infty}}
\newtheorem*{defn}{\textbf{Definition}}
\usepackage{xcolor}

\hwheader  % header for homework
%\hwslnheader   % header for homework solutions

%Comment out this line to hid "Solution: ..." boxes.
\printanswers

\begin{document}


    % \begin{defn}
    %     Let $(a_n)_{n=m}^{\infty}$ be a
    % \end{defn}

    \textbf{Definition: } Let $(a_n)_{n=m}^{\infty}$ be a sequence indexed by a set 
    $\{n \in \Z \mid  n \geq m\}$ of consecutive integers. If $n \geq m$, one defines
    the \textbf{partial sum} 
    $$s_n = a_m + a_{m+1} + \dots + a_n$$
    The associated \textbf{infinite series} is the expression $\sum_{n=m}^{\infty} a_n$.
    If the sequence $(s_n)_{n=m}^{\infty}$ of partial sums converges to $s \in \R$, one 
    writes $$\sum_{n=m}^{\infty} a_n = s$$ and says that the infinite series $\sum_{n=m}^{\infty} a_n$
    \textbf{converges} to s.

    \begin{enumerate}
        \item Prove that if $\qs a_n = s$ and $\qs b_n = t$, then 
            $$\qs (a_n + b_n) = s + t$$
            \begin{solution}
                Since $\qs a_n$ converges to $s$ we know the sequence of partial sums $(s_n)_{n=m}^{\infty}$ converges to $s$. 
                Likewise since $\qs b_n$ converges to $t$ we know the sequence of partial sums $(t_n)_{n=m}^{\infty}$ converges 
                to $t$. Thus by the addition limit law 
                $$(s_n + t_n)_{n=m}^{\infty} = s + t$$ and hence by definition of series convergence $\qs (a_n + b_n) = s + t$
            \end{solution}


        \item \begin{enumerate}
            \item Prove using the $\e$-$\d$ definition of continuity that the function $g: \R \setminus \{0\} \to \R$ given $g(x) = \frac{1}{x}$ is continious at $x_0 = 3$.
            \item Let $f: \R \to \R$ be defined so that $f(x) = 1 - x^2$ if $x \geq 0$ and $f(x) = -x$ if $x < 0$.
            Prove using the $\e$-$\d$ definition of continuity that $f$ is not continious at 0.
        \end{enumerate}
            \begin{solution}
                \begin{enumerate}
                    \item Fix $\e > 0$ and $\d =\min \left\{\frac{9\e}{2}, \frac{3}{2}\right\}$. Then for $|x-3| < \d$ we have that $x > \frac{3}{2}$. Now consider 
                    $$\left| \dfrac{1}{x} - \dfrac{1}{3}\right| = 
                    \left|\dfrac{x-3}{3x}\right| < \left|\dfrac{2|x-3|}{9}\right| = \frac{2}{9} |x-3| < \frac{2}{9} \cdot \frac{9\e}{2} = \e
                    $$
                    Thus by definition $g(x) = \frac{1}{x}$ is continous at $x = 3$.
                    \item We want to show there exits some $\e > 0$ such that for all $\d > 0$ there exists some $x \in \R$ such that $|x| < \d$ and $|f(x)-1| \geq \e$. Let $\e = 1/2$. Fix $\d > 0$ then let $x = \max\{-1/2, -\d/2\}$, then for $|x| < \d$ we have that if $-1/2 > -\d/2$ 
                    \begin{align*} 
                        |f(x)-1| &= |-x - 1| \\
                                &= |\frac{1}{2} - 1| \\
                                &= |-1/2| \\
                                &= 1/2 \geq 1/2
                    \end{align*}
                    and if $-1/2 < -\d/2$ then we have 
                    \begin{align*}
                        |f(x)-1| &= |-x - 1| \\
                                &= |\d/2 - 1| \\
                                &> |1/2| \\
                                &= 1/2 \geq 1/2
                    \end{align*}
                    So in either case we have that if $|x - 0| < \d$ then $|f(x) - f(0)| \geq 1/2$, thus $f$ is not continious at 0. 
                \end{enumerate}
            \end{solution}

        \item Prove using the $\epsilon$-$\delta$ definition of continuity that if $f: \textbf{D} \to \R$ and $g: \textbf{D} \to \R$ are both continous at $x_0$, then $f + g$ is continious at $x_0$.
            
            \begin{solution}
                Fix $\e > 0$. Then there exists some $\d_1 > 0$ such that if $|x - x_0| < \d_1$ then $|f(x) - f(x_0)| < \e/2$. Likewise there exists some $\d_2 > 0$ such that if $|x - x_0| < \d_2$ then $|g(x) - g(x_0)| < \e/2$. Let $\d = \min\{\d_1, \d_2\}$ then if $|x - x_0| < \d$ 
                \begin{align*}
                    |(f+g)(x) - (f+g)(x_0)| &= |f(x) + g(x) - f(x_0) - g(x_0)| \\
                                            &= |(f(x) - f(x_0)) + (g(x) - g(x_0))| \\
                                            &\le |f(x) - f(x_0)| + |g(x) - g(x_0)| \\
                                            &<  \e/2 + \e/2 \\
                                            &= \e 
                \end{align*}
                Thus by definition, $f + g$ is continous at $x_0$
            \end{solution}


        \item \textbf{Definition: } Suppose $f: D \to \R$ is a function, $a \in \R$ and there exists $\beta > 0$ such that D contains $(a - \beta, a) \cup (a, a + \beta)$. We say that 
        $$\lim_{x \to a} f(x) = L$$
        if given any sequence $(x_n)$ with values in $D \setminus \{a\}$ such that $\lim x_n = a$ we have $\lim f(x_n) = L$.
            \begin{enumerate}
                \item Describe the domain of $\frac{x^2-4}{x-2}$ and prove that $\lim_{x \to 2} \frac{x^2-4}{x-2} = 4$
                \item Prove that if $f: D \to \R$ and 
                $g: D \to \R$ are functions, $a \in D$,
                $\lim_{x \to a} f(x) = L$, and $\lim_{x \to a} g(x) = M$, then $\lim_{x \to a} (f+g)(x) = L + M$
            \end{enumerate}
                \begin{solution}
                    \begin{enumerate}
                        \item Domain $D = \{x \mid x \in (-\infty, 2) \cup (2, \infty)\}$. Let $(x_n)$ be a sequence in $D$ such that $\lim_{n \to \infty} x_n = 2$. Let $f(x) = \frac{x^2 - 4}{x - 2}$ where $x \in D$. Then we have that 
                        $$\lim f(x_n) = \lim_{n \to \infty} \frac{x_n^2 - 4}{x_n-2} = \lim_{n \to \infty} \frac{(x_n - 2) (x_n+2)}{x_n-2} = 
                        \lim_{n \to \infty} x_n + 2 = 2 + 2 = 4
                        $$
                        \item Let $(x_n)$ be a sequence in $D$ such that $\lim_{n \to \infty} x_n = a$. Since $\lim_{x \to a} f(x) = L$ then 
                        $\lim f(x_n) = L$ and since $\lim_{x \to a} g(x) = M$ then $\lim g(x_n) = M$. Thus we get that 
                        $$\lim (f+g)(x_n) = \lim f(x_n) + \lim g(x_n) = L + M$$ thus 
                        $$\lim_{x \to a} (f+g)(x) = L + M$$
                    \end{enumerate}
                \end{solution}
    \end{enumerate}

\end{document}  