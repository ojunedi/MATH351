\documentclass[11pt]{exam}
\usepackage[margin=1in]{geometry}
\usepackage{amsfonts, amsmath, amssymb, amsthm}
\usepackage{mathtools}
\usepackage{enumerate}
\usepackage{listings}
\usepackage{array}

\usepackage{fullpage} % margin formatting
\usepackage{enumitem} % configure enumerate and itemize
\usepackage{amsmath, amsfonts, amssymb} % math symbols
\usepackage{xcolor, colortbl} % colors, including in tables
\usepackage{makecell} % thicker \Xhline in table
\usepackage{graphicx} % images, resizing
\usepackage[T1]{fontenc}
% sometimes needed packages
% \usepackage{hyperref} % hyperlinks
% \usepackage{logicproof} % natural deduction
\usepackage{tikz} % drawing graphs
\usepackage{algpseudocode} % pseudocode

% paragraph formatting
\setlength{\parskip}{6pt}
\setlength{\parindent}{0cm}

% newline after Solution:
\renewcommand{\solutiontitle}{\noindent\textbf{Solution:}\par\noindent}
\newcommand{\sntk}{\sum_{k=1}^{n}}

% less space before itemize/enumerate
\setlist{topsep=0pt}

% creates \filcl to grey out cells for groupwork grading
\newcommand{\filcl}{\cellcolor{gray!25}}


% creates \probnum to get the problem number
\newcounter{probnumcount}
\setcounter{probnumcount}{1}
\newcommand{\probnum}{\arabic{probnumcount}. \addtocounter{probnumcount}{1}}

% use roman numerals by default
% \setlist[enumerate]{label={(\roman*)}}

% creates custom list environments for grading guidelines, question parts
\newlist{guidelines}{itemize}{1}
\setlist[guidelines]{label={}, left=0pt .. \parindent, nosep}
\newlist{gwguidelines}{enumerate}{1}
\setlist[gwguidelines]{label={(\roman*)}, nosep}
\newlist{qparts}{enumerate}{2}
\setlist[qparts]{label={(\alph*)}}
\newlist{qsubparts}{enumerate}{2}
\setlist[qsubparts]{label={(\roman*)}}
\newlist{stmts}{enumerate}{1}
\setlist[stmts]{label={(\roman*)}, nosep}



% in order to compile this file you need to get 'header.tex' from
% Canvas and change the line below to the appropriate file path
\input{"/Users/omerjunedi/Documents/math351/individual_hw/header.tex"}

% TODO: always change
\newcommand{\hwnum}{2}
\newcommand{\duedate}{Sept 6}
\usepackage{xcolor}

\hwheader  % header for homework
%\hwslnheader   % header for homework solutions

%Comment out this line to hid "Solution: ..." boxes.
\printanswers

\begin{document}

    \begin{enumerate}
        \item Suppose that \( f : \mathbb{R} \to \mathbb{R} \) and \( g : \mathbb{R} 
        \to \mathbb{R} \) are differentiable functions. Suppose also that \( f(0) = g
        (0) \) and that \( f'(x) \leq g'(x) \) for all \( x \geq 0 \). Prove that \( f(x) \leq g
        (x) \) for all \( x \geq 0 \).
            \begin{enumerate}
                \item Prove that if \( f \) is continuous on \( [a, b] \), \( f \) is 
                differentiable on \( (a, b) \), and \( f'(x) = 0 \) if \( a < x < b \), then \( 
                    f \) is constant on \( [a, b] \) (i.e., there exists \( c \in \mathbb{R} \) 
                    such that \( f(x) = c \) for all \( x \in [a, b] \)).
                \item Suppose that \( f \) and \( g \) are functions which are both continuous 
                on \( [a, b] \) and differentiable on \( (a, b) \). Prove that if \( f'(x) = g'
                (x) \) for all \( x \in (a, b) \), then there exists \( c \in \mathbb{R} \) 
                such that \( f(x) = g(x) + c \) for all \( x \in (a, b) \).
            \end{enumerate}
                \begin{solution}
                    Let $h(x) := g(x) - f(x)$. Then $h'(x) = g'(x) - f'(x) \geq 0$. Thus $h$ 
                    is an increasing function (as we proved on the last HW). Then for all $x \geq 0$ we have that 
                    $$h(x) \geq h(0) = g(0) - f(0) = 0$$
                    $$g(x) - f(x) \geq 0 \Longrightarrow f(x) \leq g(x)$$ 
                    \begin{enumerate}
                        \item Let $x_1, x_2 \in (a,b)$ such that $x_1 \neq x_2$. Then by the Mean Value Theorem there exists some $d \in (x_1, x_2)$ such that 
                        $$f'(d) = \frac{f(x_2) - f(x_1)}{x_2 - x_1} = 0 \Longrightarrow f(x_2) = f(x_1)$$
                        We have shown that two arbitrary points in $(a,b)$ are equal thus $f$
                        is a constant function over (a, b) and there exists \( c \in \mathbb{R} \) 
                        such that \( f(x) = c \) for all \( x \in (a, b) \). Now we must show that the endpoints $f(a) = f(b) = c$. Use continuity argument.
                        \item Let $h(x) := f(x) - g(x)$ for all $x \in (a, b)$ then $h'(x) = f'(x) - g'(x) = 0$. By part (a) we know that there exists some $c \in \R$ such that $h(x) = c$ for all $x \in (a, b)$. Plugging back into definition for $h$ we get $c = f(x) - g(x)$ which rearranging yields $f(x) = g(x) + c$ for all $x \in (a, b)$ as desired.
                    \end{enumerate}
                \end{solution}
        \item 
            \begin{enumerate}
                \item Suppose \( S \) is a non-empty bounded set in \( \mathbb{R} \) and that \( c \geq 0 \). We define the set \( cS \) as follows: \( cS = \{cs \,|\, s \in S\} \).
                Prove that \( \sup(cS) = c \sup(S) \). Notice that one may similarly prove that \( \inf(cS) = c \inf(S) \) (but you don't need to write down a proof of this).
                \item Suppose \( f : [a, b] \rightarrow \mathbb{R} \) is a bounded function. Prove that if \( c \geq 0 \), then
    
                \( U(cf, P) = c U(f, P) \) and \( L(cf, P) = c L(f, P) \)
                
                for any partition \( P \) of \( [a, b] \).
                \item Prove that if \( f : [a, b] \rightarrow \mathbb{R} \) is a bounded function and \( c \geq 0 \), then \( U(cf) = cU(f) \) and \( L(cf) = cL(f) \).
                \item Prove that if \( f \) is integrable and \( c \geq 0 \), then \( cf \) is integrable and
                \( \int_{a}^{b} cf(x) \,dx = c \int_{a}^{b} f(x) \,dx \).
            \end{enumerate}
                \begin{solution}
                    \begin{enumerate}
                        \item We know for all $s \in S$
                            \begin{align*}
                                s &\leq \sup(S) \\
                                cs &\leq c\sup(S) 
                            \end{align*}
                            Thus $c\sup(S)$ is an upper bound for $cS$, thus since the supremum is the least upper bound we get $\sup(cS) \leq c\sup(S)$. Likewise we know that 
                            \begin{align*}
                                cs &\leq \sup(cS) \\
                                s &\leq \frac{\sup(cS)}{c}
                            \end{align*}
                            Thus $\frac{\sup(cS)}{c}$ is an upper bound for the set $S$, then since the supremum is the least upper bound we get that $\sup(S) \leq \frac{\sup(cS)}{c}$ and thus we get $c\sup(S) \leq \sup(cS)$. Combining the fact that $\sup(cS) \leq c\sup(S)$ and  $c\sup(S) \leq \sup(cS)$, we get the result $\sup(cS) = c\sup(S)$ as desired.
                        \item Let $P = \{t_0, t_1, t_2, \ldots , t_n\}$ be a partition of $[a, b]$ and $c \geq 0$. Then 
                            \begin{align*}
                                U(cf, P) &= \sntk \sup\{cf([t_{k-1}, t_k])\} \cdot (t_k - t_{k-1}) \\
                                        &= c \sntk \sup\{f([t_{k-1}, t_k])\} \cdot (t_k - t_{k-1}) \\
                                        &= c \cdot U(f, P)
                            \end{align*}
                            Likewise
                            \begin{align*}
                                L(cf, P) &= \sntk \inf\{cf([t_{k-1}, t_k])\} \cdot (t_k - t_{k-1}) \\
                                        &= c \sntk \inf\{f([t_{k-1}, t_k])\} \cdot (t_k - t_{k-1}) \\
                                        &= c \cdot L(f, P)
                            \end{align*}
                        \item 
                        \begin{align*}
                            U(cf) &= \inf\{U(cf, P)\} \\
                                 &= \inf\{cU(f, P)\} \\
                                 &= c \inf\{U(f, P)\} \\
                                 &= c U(f)
                        \end{align*}
                        Likewise
                        \begin{align*}
                            L(cf) &= \sup\{L(cf, P)\} \\
                                 &= \sup\{cL(f, P)\} \\
                                 &= c \sup\{L(f, P)\} \\
                                 &= c L(f)
                        \end{align*}
                        \item Assume $f$ is integrable, then we know by using part (c)
                        \begin{align*}
                            U(f) &= L(f) \\
                            cU(f) &= cL(f) \\
                            U(cf) &= L(cf)
                        \end{align*}
                        Thus $cf$ is integrable and $c\int_{a}^{b} f  = \int_{a}^{b} cf$.
                    \end{enumerate}
                \end{solution}
        \item Suppose that \( f : [a, b] \rightarrow \mathbb{R} \) is bounded on \( [a, b] \).
            \begin{enumerate}
                \item Prove that \( U(-f) = -L(f) \) and \( L(-f) = -U(f) \). Hint: You may use the fact that if \( S \) is a nonempty subset of \( \mathbb{R} \), then \( \inf S = -\sup(-S) \). You do not need to justify this fact.
                \item Prove that if \( f \) is integrable on \( [a, b] \), then \( -f \) is integrable on \( [a, b] \) and
    
                \( \int_{a}^{b} -f(x) \,dx = -\int_{a}^{b} f(x) \,dx \).
                \item Conclude that if \( c \in \mathbb{R} \) and \( f \) is integrable on \( [a, b] \), then \( cf \) is integrable on \( [a, b] \) and
    
                \( \int_{a}^{b} cf(x) \,dx = c \int_{a}^{b} f(x) \,dx \).
            \end{enumerate}
                \begin{solution}
                    \begin{enumerate}
                        \item Let $P = \{t_0, t_1, t_2, \ldots , t_n\}$ be a partition of $[a, b]$. Then 
                            \begin{align*}
                                U(-f) &= \inf\{U(-f, P)\} \\
                                    &= \inf\left(\sntk \sup\{-f([t_{k-1}, t_k])\} \cdot (t_k - t_{k-1})\right) \\
                                    &= \inf\left(\sntk -\inf\{f([t_{k-1}, t_k])\} \cdot (t_k - t_{k-1})\right) \\
                                    &= -\sup\left(\sntk \inf\{f([t_{k-1}, t_k])\} \cdot (t_k - t_{k-1})\right) \\
                                    &= -\sup\{L(f, P)\}  \\
                                    &= -L(f)
                            \end{align*}
                            and 
                            \begin{align*}
                                L(-f) &= \sup\{L(-f, P)\} \\
                                    &= \sup\left(\sntk \inf\{-f([t_{k-1}, t_k])\} \cdot (t_k - t_{k-1})\right) \\
                                    &= -\inf\left(\sntk -\inf\{-f([t_{k-1}, t_k])\} \cdot (t_k - t_{k-1})\right) \\
                                    &= -\inf\left(-\sntk -\sup\{f([t_{k-1}, t_k])\} \cdot (t_k - t_{k-1})\right) \\
                                    &= -\inf\left(\sntk \sup\{f([t_{k-1}, t_k])\} \cdot (t_k - t_{k-1})\right) \\
                                    &= -\inf\{U(f, P)\}  \\
                                    &= -U(f)
                            \end{align*}
                        \item This is a specific case of 2 part (d) with $c = -1$.
                        \item 3c solution
                    \end{enumerate}
                \end{solution}
            \end{enumerate}
            


\end{document}