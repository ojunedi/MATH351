\documentclass[11pt]{exam}
\usepackage[margin=1in]{geometry}
\usepackage{amsfonts, amsmath, amssymb, amsthm}
\usepackage{mathtools}
\usepackage{enumerate}
\usepackage{listings}
\usepackage{array}

\usepackage{fullpage} % margin formatting
\usepackage{enumitem} % configure enumerate and itemize
\usepackage{amsmath, amsfonts, amssymb} % math symbols
\usepackage{xcolor, colortbl} % colors, including in tables
\usepackage{makecell} % thicker \Xhline in table
\usepackage{graphicx} % images, resizing
\usepackage[T1]{fontenc}
% sometimes needed packages
% \usepackage{hyperref} % hyperlinks
% \usepackage{logicproof} % natural deduction
\usepackage{tikz} % drawing graphs
\usepackage{algpseudocode} % pseudocode

% paragraph formatting
\setlength{\parskip}{6pt}
\setlength{\parindent}{0cm}

% newline after Solution:
\renewcommand{\solutiontitle}{\noindent\textbf{Solution:}\par\noindent}

% less space before itemize/enumerate
\setlist{topsep=0pt}

% creates \filcl to grey out cells for groupwork grading
\newcommand{\filcl}{\cellcolor{gray!25}}


% creates \probnum to get the problem number
\newcounter{probnumcount}
\setcounter{probnumcount}{1}
\newcommand{\probnum}{\arabic{probnumcount}. \addtocounter{probnumcount}{1}}

% use roman numerals by default
% \setlist[enumerate]{label={(\roman*)}}

% creates custom list environments for grading guidelines, question parts
\newlist{guidelines}{itemize}{1}
\setlist[guidelines]{label={}, left=0pt .. \parindent, nosep}
\newlist{gwguidelines}{enumerate}{1}
\setlist[gwguidelines]{label={(\roman*)}, nosep}
\newlist{qparts}{enumerate}{2}
\setlist[qparts]{label={(\alph*)}}
\newlist{qsubparts}{enumerate}{2}
\setlist[qsubparts]{label={(\roman*)}}
\newlist{stmts}{enumerate}{1}
\setlist[stmts]{label={(\roman*)}, nosep}



% in order to compile this file you need to get 'header.tex' from
% Canvas and change the line below to the appropriate file path
%%% theorems

\theoremstyle{plain}            % following are "theorem" style
\newtheorem{theorem}{Theorem}[section]
\newtheorem{lemma}[theorem]{Lemma}
\newtheorem{corollary}[theorem]{Corollary}
\newtheorem{proposition}[theorem]{Proposition}
\newtheorem{claim}[theorem]{Claim}
\newtheorem{fact}[theorem]{Fact}
\newtheorem{openproblem}[theorem]{Open Problem}

\theoremstyle{definition}       % following are def style
\newtheorem{definition}[theorem]{Definition}
\newtheorem{conjecture}[theorem]{Conjecture}
\newtheorem{example}[theorem]{Example}
\newtheorem{protocol}[theorem]{Protocol}
\newtheorem{exercise}[theorem]{Exercise}

\theoremstyle{remark}           % following are remark style
\newtheorem{remark}[theorem]{Remark}
\newtheorem{note}[theorem]{Note}
\newtheorem*{aside}{Aside}
%\newtheorem*{solution}{Solution}
\newtheorem*{gn}{Grading Note} % grading note

%%% special sets
\newcommand{\bit}{\ensuremath{\{0,1\}}}
\newcommand{\bitt}{\ensuremath{\{-1,1\}}}
\newcommand{\ball}{\ensuremath{\mathcal{B}}}
\newcommand{\sph}{\ensuremath{\mathbb{S}}}
\newcommand{\odisc}[2]{\ensuremath{D(#1, #2)}}
\newcommand{\cdisc}[2]{\ensuremath{\bar{D}(#1, #2)}}
\newcommand{\emp}{\varnothing}

% constants
\newcommand{\E}{\ensuremath{\mathrm{e}}}
\newcommand{\I}{\ensuremath{\mathrm{i}}}
\newcommand{\Id}{\ensuremath{\mathrm{I}}}
\newcommand{\paulix}{\ensuremath{\mathrm{X}}}
\newcommand{\pauliy}{\ensuremath{\mathrm{Y}}}
\newcommand{\pauliz}{\ensuremath{\mathrm{Z}}}

% font for general-purpose algorithms
\newcommand{\algo}[1]{\ensuremath{\mathsf{#1}}}
% font for general-purpose computational problems
\newcommand{\problem}[1]{\ensuremath{\mathsf{#1}}}
% font for complexity classes
\newcommand{\class}[1]{\ensuremath{\mathsf{#1}}}

% asymptotics
\DeclareMathOperator{\poly}{poly}
\DeclareMathOperator{\polylog}{polylog}
\DeclareMathOperator{\negl}{negl}
\DeclareMathOperator{\bigO}{O}
\DeclareMathOperator{\litO}{o}
\DeclareMathOperator{\Otil}{\tilde{O}}
\DeclareMathOperator{\Ostar}{O^*}

%%% "LEFT-RIGHT" PAIRS OF SYMBOLS

% inner product
\DeclarePairedDelimiter\inner{\langle}{\rangle}
% absolute value
\DeclarePairedDelimiter\abs{\lvert}{\rvert}
% a set
\DeclarePairedDelimiter\set{\{}{\}}
% parens
\DeclarePairedDelimiter\parens{(}{)}
% tuple, alias for parens
\DeclarePairedDelimiter\tuple{(}{)}
% square brackets
\DeclarePairedDelimiter\bracks{[}{]}
% rounding off
\DeclarePairedDelimiter\round{\lfloor}{\rceil}
% floor function
\DeclarePairedDelimiter\floor{\lfloor}{\rfloor}
% ceiling function
\DeclarePairedDelimiter\ceil{\lceil}{\rceil}
% length of some vector, element
\DeclarePairedDelimiter\length{\lVert}{\rVert}
% "lifting" of a residue class
\DeclarePairedDelimiter\lift{\llbracket}{\rrbracket}
\DeclarePairedDelimiter\len{\lvert}{\rvert}
% bra-kets
\DeclarePairedDelimiter\bra{\langle}{\rvert}
\DeclarePairedDelimiter\ket{\lvert}{\rangle}
\newcommand{\braket}[2]{\ensuremath{\langle #1 \vert #2 \rangle}}
\newcommand{\ketbra}[2]{\ensuremath{\lvert #1 \rangle \langle #2 \rvert}}

%%% spacing

\newcommand{\ws}{\hspace{1pt}}
\newcommand{\wws}{\hspace{2pt}}
\newcommand{\hs}{\hspace{4pt}}
\newcommand{\hhs}{\hspace{8pt}}
\newcommand{\hhhs}{\hspace{12pt}}

%%% LISTS

\newcommand{\oneto}{1, \ldots,}
\newcommand{\onetop}{1 \cdots,}
\newcommand{\zeroto}{0, \ldots,}
\newcommand{\zerotop}{0 \cdots,}
\newcommand{\perm}[1]{\mathbf{(#1)}}
\newcommand{\permv}[1]{(#1)}
\newcommand{\varind}[2]{#1_1, \ldots, #1_#2}
\newcommand{\varindz}[2]{#1_0, \ldots, #1_#2}
\newcommand{\varindp}[2]{#1_1 \cdots #1_#2}
\newcommand{\varindpz}[2]{#1_0 \cdots #1_#2}
\newcommand{\seq}[2]{(#1_#2)_{#2=1}^\infty}
\newcommand{\seqz}[2]{(#1_#2)_{#2=0}^\infty}

%%% MATH OPERATORS
%\DeclareMathOperator{\Ex}{\mathbf{E}}
%\DeclareMathOperator{\Pr}{Pr}
\DeclareMathOperator{\Var}{Var}
\DeclareMathOperator{\Span}{Span}
\DeclareMathOperator{\tr}{Tr}
\DeclareMathOperator{\supp}{Supp}
\DeclareMathOperator{\im}{Im}
\DeclareMathOperator{\var}{var}
\DeclareMathOperator{\vol}{vol}
\DeclareMathOperator{\sign}{sign}
\DeclareMathOperator{\dkl}{D_{KL}}
\DeclareMathOperator{\entr}{H}
\DeclareMathOperator{\fid}{F}
\DeclareMathOperator{\dist}{D}
\DeclareMathOperator{\ad}{ad}

% hats

\newcommand{\fhat}{\ensuremath{\hat{f}}}
\newcommand{\phat}{\ensuremath{\hat{p}}}
\newcommand{\that}{\ensuremath{\hat{t}}}

%%% BLACKBOARD SYMBOLS

\newcommand{\C}{\ensuremath{\mathbb{C}}}
\newcommand{\D}{\ensuremath{\mathbb{D}}}
\newcommand{\F}{\ensuremath{\mathbb{F}}}
\newcommand{\G}{\ensuremath{\mathbb{G}}}
\newcommand{\J}{\ensuremath{\mathbb{J}}}
\newcommand{\N}{\ensuremath{\mathbb{N}}}
\newcommand{\Q}{\ensuremath{\mathbb{Q}}}
\newcommand{\R}{\ensuremath{\mathbb{R}}}
\newcommand{\T}{\ensuremath{\mathbb{T}}}
\newcommand{\Z}{\ensuremath{\mathbb{Z}}}
\newcommand{\QR}{\ensuremath{\mathbb{QR}}}
\newcommand*{\e}{\varepsilon}
\renewcommand{\d}{\delta}

% sets in calligraphic type

\newcommand{\calD}{\ensuremath{\mathcal{D}}}
\newcommand{\calF}{\ensuremath{\mathcal{F}}}
\newcommand{\calG}{\ensuremath{\mathcal{G}}}
\newcommand{\calH}{\ensuremath{\mathcal{H}}}
\newcommand{\calI}{\ensuremath{\mathcal{I}}}
\newcommand{\calL}{\ensuremath{\mathcal{L}}}
\newcommand{\calN}{\ensuremath{\mathcal{N}}}
\newcommand{\calP}{\ensuremath{\mathcal{P}}}
\newcommand{\calS}{\ensuremath{\mathcal{S}}}
\newcommand{\calX}{\ensuremath{\mathcal{X}}}
\newcommand{\calY}{\ensuremath{\mathcal{Y}}}

% matrices and vectors

\newcommand{\matA}{\ensuremath{\mathbf{A}}}
\newcommand{\matB}{\ensuremath{\mathbf{B}}}
\newcommand{\matC}{\ensuremath{\mathbf{C}}}
\newcommand{\matD}{\ensuremath{\mathbf{D}}}
\newcommand{\matE}{\ensuremath{\mathbf{E}}}
\newcommand{\matF}{\ensuremath{\mathbf{F}}}
\newcommand{\matG}{\ensuremath{\mathbf{G}}}
\newcommand{\matH}{\ensuremath{\mathbf{H}}}
\newcommand{\matI}{\ensuremath{\mathbf{I}}}
\newcommand{\matJ}{\ensuremath{\mathbf{J}}}
\newcommand{\matK}{\ensuremath{\mathbf{K}}}
\newcommand{\matL}{\ensuremath{\mathbf{L}}}
\newcommand{\matM}{\ensuremath{\mathbf{M}}}
\newcommand{\matN}{\ensuremath{\mathbf{N}}}
\newcommand{\matO}{\ensuremath{\mathbf{O}}}
\newcommand{\matP}{\ensuremath{\mathbf{P}}}
\newcommand{\matQ}{\ensuremath{\mathbf{Q}}}
\newcommand{\matR}{\ensuremath{\mathbf{R}}}
\newcommand{\matS}{\ensuremath{\mathbf{S}}}
\newcommand{\matT}{\ensuremath{\mathbf{T}}}
\newcommand{\matU}{\ensuremath{\mathbf{U}}}
\newcommand{\matV}{\ensuremath{\mathbf{V}}}
\newcommand{\matW}{\ensuremath{\mathbf{W}}}
\newcommand{\matX}{\ensuremath{\mathbf{X}}}
\newcommand{\matY}{\ensuremath{\mathbf{Y}}}
\newcommand{\matZ}{\ensuremath{\mathbf{Z}}}
\newcommand{\matzero}{\ensuremath{\mathbf{0}}}

\newcommand{\veca}{\ensuremath{\mathbf{a}}}
\newcommand{\vecb}{\ensuremath{\mathbf{b}}}
\newcommand{\vecc}{\ensuremath{\mathbf{c}}}
\newcommand{\vecd}{\ensuremath{\mathbf{d}}}
\newcommand{\vece}{\ensuremath{\mathbf{e}}}
\newcommand{\vecf}{\ensuremath{\mathbf{f}}}
\newcommand{\vecg}{\ensuremath{\mathbf{g}}}
\newcommand{\vech}{\ensuremath{\mathbf{h}}}
\newcommand{\veck}{\ensuremath{\mathbf{k}}}
\newcommand{\vecm}{\ensuremath{\mathbf{m}}}
\newcommand{\vecp}{\ensuremath{\mathbf{p}}}
\newcommand{\vecq}{\ensuremath{\mathbf{q}}}
\newcommand{\vecr}{\ensuremath{\mathbf{r}}}
\newcommand{\vecs}{\ensuremath{\mathbf{s}}}
\newcommand{\vect}{\ensuremath{\mathbf{t}}}
\newcommand{\vecu}{\ensuremath{\mathbf{u}}}
\newcommand{\vecv}{\ensuremath{\mathbf{v}}}
\newcommand{\vecw}{\ensuremath{\mathbf{w}}}
\newcommand{\vecx}{\ensuremath{\mathbf{x}}}
\newcommand{\vecy}{\ensuremath{\mathbf{y}}}
\newcommand{\vecz}{\ensuremath{\mathbf{z}}}
\newcommand{\veczero}{\ensuremath{\mathbf{0}}}
\newcommand{\vecone}{\ensuremath{\mathbf{1}}}

\newcommand{\vecell}{\ensuremath{\boldsymbol\ell}}
\newcommand{\vecalpha}{\ensuremath{\boldsymbol\alpha}}
\newcommand{\vecbeta}{\ensuremath{\boldsymbol\beta}}
\newcommand{\veceta}{\ensuremath{\boldsymbol\eta}}
\newcommand{\vecmu}{\ensuremath{\boldsymbol\mu}}
\newcommand{\vecphi}{\ensuremath{\boldsymbol\phi}}
\newcommand{\vecsigma}{\ensuremath{\boldsymbol\sigma}}
\newcommand{\vectheta}{\ensuremath{\boldsymbol\theta}}
\newcommand{\vecxi}{\ensuremath{\boldsymbol\xi}}

%%% misc

\newcommand{\ind}{\ensuremath{\mathbf{1}}}

\newcommand{\congmod}[3]{#1 \equiv #2 \textrm{ modulo } #3}

\newcommand{\dee}{\,\mathrm{d}}
\newcommand{\de}{\mathrm{d}}
\newcommand{\dx}{\,\mathrm{d} x}

\newcommand{\ol}{\overline}
\newcommand{\inv}[1]{\ensuremath{#1^{-1}}}
\newcommand{\tsp}[1]{\ensuremath{#1^{\top}}}


\newcommand{\eps}{\varepsilon}
\newcommand{\ph}{\varphi}

\newcommand{\Ra}{\Rightarrow}
\newcommand{\Lra}{\Leftrightarrow}
\newcommand{\rsqa}{\rightsquigarrow}

\newcommand{\trl}{\triangleleft}
\newcommand{\trr}{\triangleright}

\newcommand{\func}[3]{#1: #2 \to #3}
\newcommand{\dd}[1]{\frac{\mathrm{d}}{\mathrm{d}#1}}
\newcommand{\ptl}[1]{\frac{\partial}{\partial #1}}
\newcommand{\prtl}[2]{\frac{\partial #1}{\partial #2}}

\newcommand{\matrixtt}[4]{
  \begin{pmatrix*}[r]
        #1 & #2 \\
        #3 & #4
    \end{pmatrix*}
}

%%% for homework and section notes
\newcommand{\blue}[1]{\textcolor{blue}{#1}}

\newcommand{\commonheader}[2]{
    \pagestyle{headandfoot}
    \setlength{\headheight}{26pt}
    \setlength{\headsep}{16pt}

    \header
        {\small{\textbf{MATH 351: Principles of Analysis}} \\ \footnotesize{\textbf{University of Michigan, Winter 2024}}}
        {#1}
        {#2}

    \firstpageheadrule
    \runningheadrule

    \footer
        {}
        {\thepage}
        {}
}

\newcommand{\hwheader}{
    \commonheader
        {\Large \textbf{HW \hwnum}}
        {}
}

\newcommand{\hwslnheader}{
    \commonheader
    	{}
        {\Large \textbf{Solutions to Homework \hwnum}}
    \printanswers
}

\newcommand{\notesheader}{
    \commonheader
    	{}
        {\Large \textbf{Discussion Notes \sectionnum}}
}

\newcommand{\practiceheader}{
    \commonheader
    	{}
        {\Large \textbf{Discussion Worksheet \sectionnum}}
}

\newcommand{\practiceslnheader}{
    \commonheader
    	{}
        {\Large \textbf{Solutions to Discussion Worksheet \sectionnum}}
}

\newcommand{\reviewheader}{
    \commonheader 
    \smallskip
    	{}
        {\Large \textbf{Midterm Review Notes}}
}

\newcommand{\hwpreface}{
\noindent We may grade a \textbf{subset of the assigned questions}, to be determined after the deadline, so that we can provide better feedback on the graded questions.

\noindent Unless otherwise stated, each question requires sufficient justification to convince the reader of the correctness of your answer.

\noindent For bonus questions, we will not provide any insight during office hours or Piazza, and we do not guarantee anything about the difficulty of these questions.
 
\noindent We strongly encourage you to typeset your solutions in \LaTeX.

\noindent If you collaborated with someone, you must state their name(s). You must write your own solution for all problems and may not look at any other student’s write-up.
}
% \newcommand{\hwpreface}{
% \noindent Problems marked with \textbf{E} are graded on effort, which means that they are graded subjectively on the perceived effort you put into them, rather than on correctness. For bonus questions, we will not provide any insight during office hours or Piazza, and we do not guarantee anything about the difficulty of these questions. \vspace{1ex}\\
% Problems marked with a \textbf{G} are group problems. A group of 1-3 students may work on these together and turn in one assignment for the entire group. Each member listed should have made real contributions. The group problems will be turned in separately on gradescope. \vspace{1ex}\\
% If you collaborated with someone, you must state their names. With the exception of group problems, you must write your own solution and may not look at any other student’s write-up. \vspace{1ex}\\
% We strongly encourage you to typeset your solutions in \LaTeX.
% }


\newcommand{\hint}[1]{
\emph{Hint}: #1
}
\newcommand{\bonus}{
\emph{Optional bonus}:
}
\newcommand{\extracredit}{
\emph{Extra credit}: 
}

% for effort and group questions
\let\Eitem=\relax
\let\Gitem=\relax
\def\effortE{\textbf{P}~}
\def\groupG{\textbf{G}~}
\makeatletter
\def\Eitem{%
    \expandafter\let\expandafter\originallabel\csname labelenum\romannumeral\@enumdepth\endcsname
    \expandafter\def\csname labelenum\romannumeral\@enumdepth\expandafter\endcsname\expandafter{%
        \expandafter\effortE\originallabel}%
    \item
    \expandafter\let\csname labelenum\romannumeral\@enumdepth\endcsname\originallabel
}

\def\Gitem{%
    \expandafter\let\expandafter\originallabel\csname labelenum\romannumeral\@enumdepth\endcsname
    \expandafter\def\csname labelenum\romannumeral\@enumdepth\expandafter\endcsname\expandafter{%
        \expandafter\groupG\originallabel}%
    \item
    \expandafter\let\csname labelenum\romannumeral\@enumdepth\endcsname\originallabel
}

\def\EGitem{%
    \expandafter\let\expandafter\originallabel\csname labelenum\romannumeral\@enumdepth\endcsname
    \expandafter\def\csname labelenum\romannumeral\@enumdepth\expandafter\endcsname\expandafter{%
        \expandafter\effortE\groupG\originallabel}%
    \item
    \expandafter\let\csname labelenum\romannumeral\@enumdepth\endcsname\originallabel
}
\makeatother

\allowdisplaybreaks


% TODO: always change
\newcommand{\hwnum}{4-1}
\newcommand{\duedate}{Jan 29}
\usepackage{xcolor}

\hwheader  % header for homework
%\hwslnheader   % header for homework solutions

%Comment out this line to hid "Solution: ..." boxes.
\printanswers


\begin{document}


    \begin{enumerate}
        \item Prove that if $f: \R \to \R$ and $g: \R \to \R$ are differentiable at $x_0 \in \R$, then 
        $fg$ is differentiable at $x_0$ and $(fg)'(x_0) = f'(x_0)g(x_0) + g'(x_0)f(x_0).$
            \begin{solution}
                We know that $f$ and $g$ are differentiable at $x_0$ thus 
                $$f'(x_0)  = \lim_{x \to x_0} \frac{f(x) - f(x_0)}{x - x_0} 
                \displayand g'(x_0)  = \lim_{x \to x_0} \frac{g(x) - g(x_0)}{x - x_0}$$
                Let $(x_n)$ be a sequence in $\R \setminus \{x_0\}$ such that $\lim_{n \to \infty} x_n = x_0$. Then by definition of the limit 
                we get that 
                $$f'(x_0) = \lim_{n \to \infty} \frac{f(x_n) - f(x_0)}{x_n - x_0} \displayand g'(x_0) = \lim_{n \to \infty} \frac{g(x_n) - g(x_0)}{x_n - x_0} $$ and 
                $$$$
                Now consider 
                \begin{align*}
                    \lim_{n \to \infty} \frac{(fg)(x_n) - (fg)(x_0)}{x - x_0} 
                    &= \lim_{n \to \infty} \frac{f(x_n)g(x_n) - f(x_0)g(x_0)}{x-x_0}  \\
                    &= \lim_{n \to \infty} \frac{f(x_n)g(x_n) - f(x_n)g(x_0) + f(x_n)g(x_0) -f(x_0)g(x_0)}{x-x_0} \\
                    &= \lim_{n \to \infty} \frac{f(x_n)[g(x_n) - g(x_0)] + g(x_0)[f(x_n) - f(x_0)]}{x-x_0} \\
                    &=  \lim_{n \to \infty} \frac{f(x_n)[g(x_n) - g(x_0)]}{x-x_0} + \lim_{n \to \infty} \frac{g(x_0)[f(x_n) - f(x_0)]}{x-x_0} \\
                    &= \lnti f(x_n) \cdot \lnti \frac{g(x_n) - g(x_0)}{x-x_0} + \lnti g(x_0) \cdot \lnti \frac{f(x_n) - f(x_0)}{x-x_0} \\
                    &= f(x_0)g'(x_0) + g(x_0)f'(x_0)
                \end{align*}
                Thus the by definition of the derivative $(fg)'(x_0) = f'(x_0)g(x_0) + g'
                (x_0)f(x_0)$
            \end{solution}
            \break
        \item 
            \begin{enumerate}
                \item Prove that the function $f: \R \setminus \{\frac{1}{2}\} \to \R$ given by $f(x) = \frac{3x+4}{2x-1}$ is differentiable at 
                $x_0 = 1$ and evaluate $f'(1)$
                \item Prove that if $g: \R \to \R$ given by $x^{1/3}$ is not differentiable at $x_0 = 0$ 
            \end{enumerate}
                \begin{solution}
                    \begin{enumerate}
                        \item 
                        Let $(x_n)$ be a sequence in $\R \setminus \{\frac{1}{2}\}$ such that $\lnti x_n = x_0$. Then 
                        \begin{align*}
                            f'(x_0) &= \lnti \frac{f(x_n) - f(x_0)}{x_n - x_0} \\ 
                            &= \lnti \frac{\frac{3x_n + 4}{2x_n -1} -\frac{3x_0 + 4}{2x_0-1}}{x_n-x_0} \\
                            &= \lnti \frac{11(x_0 - x_n)}{(2x_n-1)(2x_0-1)(x_n-x_0)} \\
                            &= \lnti - \frac{11}{(2x_n-1)(2x_0-1)} \\
                            &= \frac{\lnti -11}{\lnti (2x_n - 1)(2x_0 - 1)} \\
                            &= \frac{\lnti -11}{\lnti (2x_n - 1) \cdot \lnti (2x_0 - 1)} \\
                            &= \frac{-11}{(2x_0 - 1)^2}
                        \end{align*}
                        Thus $f$ is differentiable at $x_0 = 1$.
                        $$f'(1) = \frac{-11}{(2(1) - 1)^2} = -11$$
                        \item In order to prove $f$ is not differentiable at 0, we must show some sequence in $\R \setminus \{0\}$ such that it 
                        converges to 0 and $\lnti \frac{f(x_n) - f(0)}{x_n - 0}$  does not exist. Let $x_n = 1/n$. We proved in class that 
                        $\lnti x_n = 0$. Consider
                        \begin{align*}
                            \lnti \frac{f(x_n) - f(0)}{x_n - 0} &= \lnti \frac{f(x_n)}{x_n} \\
                            &= \lnti \frac{(x_n)^{\frac{1}3{}}}{x_n} \\
                            &= \lnti (x_n)^{-\frac{2}{3}} \\
                            &= \lnti \left(\frac{1}{n}\right)^{-\frac{2}{3}} \\
                            &= \lnti n^{2/3}
                        \end{align*}
                        We proved in HW that $\lim_{n \to \infty} \frac{1}{n^p} = 0$ if $p > 0$. Thus $\lnti \frac{1}{n^{\frac{2}{3}}} = 0$. We 
                        also proved in HW that if a sequence $(x_n)$ converges to 0 and $x_n > 0$ for all $n$, then $\frac{1}{x_n}$ diverges to 
                        infinity. $\frac{1}{n^{\frac{2}{3}}} > 0$ for all $n$ and converges to $0$. Thus $\lnti n^{\frac{2}{3}}$ diverges to 
                        infinity. Thus since the limit diverges, $f$ is not differentiable at 0.
                    \end{enumerate}
                \end{solution}
            \break
        \item Prove that if $f: D \to \R$ is differentiable at a point $a \in D$ then $f$ is continious at $a$
            \begin{solution}
                Assume $f$ is differentiable. Let $(x_n)$ be a sequence in $D \setminus \{a\}$ such 
                that $\lnti x_n = a$. Then $\lnti \frac{f(x_n) - f(a)}{x_n - a}$ exists and is the 
                value of the derivative of $f$ at $a$. Now consider
                \begin{align*}
                    \lnti f(x_n) &= \lnti f(x_n) - f(a) + f(a) \\
                                &= \lnti f(x_n) - f(a) + \lnti f(a) \\
                                &= \lnti \frac{f(x_n) - f(a)}{x_n - a} \cdot (x_n -a) + f(a) \\
                                &= \lnti \frac{f(x_n) - f(a)}{x_n - a} \cdot \lnti (x_n -a) +  f(a) \\
                                &= f'(a) \cdot (a-a) + f(a) \\
                                &= 0 + f(a) \\
                                &= f(a)
                \end{align*}
                We have shown that for any sequence $(x_n)$ in $D \setminus \{a\}$ that converges 
                to $a$, the sequence $(f(x_n))$ converges to $f(a)$, thus $f$ is continous at $a$.
            \end{solution}
        \item 
            \begin{enumerate}
                \item Suppose $f: D \to \R$ is a differentiable function, that $D$ contains an open interval $(a, b)$ for some $a < b$, and that 
                $f'(x) > 0$ for all $x \in (a, b)$. Prove that $f$ is strictly increasing on $(a, b)$. That is, prove that if $a < x < y < b$, 
                then $f(x) < f(y)$.
                \item Suppose $f: D \to \R$ is a differentiable function, that $D$ contains an open interval $(a, b)$ for some $a < b$, and that 
                $f'(x) < 0$ for all $x \in (a, b)$. Prove that $f$ is strictly decreasing on $(a, b)$. 
            \end{enumerate}
                \begin{solution}
                    \begin{enumerate}
                        \item Let $x, y \in (a,b)$ such that $x < y$. Note that $[x, y] \subseteq (a, b)$. Thus $f$ is differentiable over $(x, 
                        y)$ and continious over $[x, y]$. Thus we can apply we Mean Value Theorem, so there exists some $x_0 \in (x, y)$ such 
                        that 
                        $$f'(x_0) = \frac{f(y) - f(x)}{y - x}$$ rearranging we get that $$f'(x_0) \cdot (y - x) = f(y) - f(x)$$ note that $y - x 
                        > 0$ and $f'(x_0) > 0$ by assumption. Thus $f(y) - f(x) > 0$ implying that $f(y) > f(x)$. 
                        Hence $f$ is strictly increasing on $(a, b)$
                        \item Let $x, y \in (a, b)$ such that $x < y$. Note that $[x, y] \subseteq (a, b)$. Thus $f$ is differentiable over $(x, 
                        y)$ and continious over $[x, y]$. Thus we can apply we Mean Value Theorem, so there exists some $x_0 \in (x, y)$ such 
                        that $$f'(x_0) = \frac{f(y) - f(x)}{y - x}$$ rearranging we get that $$f'(x_0) \cdot (y - x) = f(y) - f(x)$$ note that 
                        $f'(x_0) < 0$ and $y - x > 0$ thus their product $f'(x_0) \cdot (y - x) < 0$, which gives us $f(y) - f(x) < 0$ implying that $f(y) < f(x)$. Hence $f$ is strictly decreasing on $(a, b)$
                    \end{enumerate}
                \end{solution}
    \end{enumerate}


\end{document}