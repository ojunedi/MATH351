\documentclass[11pt]{exam}
\usepackage[margin=1in]{geometry}
\usepackage{amsfonts, amsmath, amssymb, amsthm}
\usepackage{mathtools}
\usepackage{enumerate}
\usepackage{listings}
\usepackage{array}

\usepackage{fullpage} % margin formatting
\usepackage{enumitem} % configure enumerate and itemize
\usepackage{amsmath, amsfonts, amssymb} % math symbols
\usepackage{xcolor, colortbl} % colors, including in tables
\usepackage{makecell} % thicker \Xhline in table
\usepackage{graphicx} % images, resizing
\usepackage[T1]{fontenc}
% sometimes needed packages
% \usepackage{hyperref} % hyperlinks
% \usepackage{logicproof} % natural deduction
\usepackage{tikz} % drawing graphs
\usepackage{algpseudocode} % pseudocode

% paragraph formatting
\setlength{\parskip}{6pt}
\setlength{\parindent}{0cm}

% newline after Solution:
\renewcommand{\solutiontitle}{\noindent\textbf{Solution:}\par\noindent}

% less space before itemize/enumerate
\setlist{topsep=0pt}

% creates \filcl to grey out cells for groupwork grading
\newcommand{\filcl}{\cellcolor{gray!25}}


% creates \probnum to get the problem number
\newcounter{probnumcount}
\setcounter{probnumcount}{1}
\newcommand{\probnum}{\arabic{probnumcount}. \addtocounter{probnumcount}{1}}

% use roman numerals by default
% \setlist[enumerate]{label={(\roman*)}}

% creates custom list environments for grading guidelines, question parts
\newlist{guidelines}{itemize}{1}
\setlist[guidelines]{label={}, left=0pt .. \parindent, nosep}
\newlist{gwguidelines}{enumerate}{1}
\setlist[gwguidelines]{label={(\roman*)}, nosep}
\newlist{qparts}{enumerate}{2}
\setlist[qparts]{label={(\alph*)}}
\newlist{qsubparts}{enumerate}{2}
\setlist[qsubparts]{label={(\roman*)}}
\newlist{stmts}{enumerate}{1}
\setlist[stmts]{label={(\roman*)}, nosep}



% in order to compile this file you need to get 'header.tex' from
% Canvas and change the line below to the appropriate file path
\input{"/Users/omerjunedi/Documents/math351/group_hw/header.tex"}

% TODO: always change
\newcommand{\hwnum}{4-1}
\newcommand{\duedate}{Jan 29}
\usepackage{xcolor}

\hwheader  % header for homework
%\hwslnheader   % header for homework solutions

%Comment out this line to hid "Solution: ..." boxes.
\printanswers


\begin{document}


    \begin{enumerate}
        \item Prove that if $f: \R \to \R$ and $g: \R \to \R$ are differentiable at $x_0 \in \R$, then 
        $fg$ is differentiable at $x_0$ and $(fg)'(x_0) = f'(x_0)g(x_0) + g'(x_0)f(x_0).$
            \begin{solution}
                We know that $f$ and $g$ are differentiable at $x_0$ thus 
                $$f'(x_0)  = \lim_{x \to x_0} \frac{f(x) - f(x_0)}{x - x_0} 
                \displayand g'(x_0)  = \lim_{x \to x_0} \frac{g(x) - g(x_0)}{x - x_0}$$
                Let $(x_n)$ be a sequence in $\R \setminus \{x_0\}$ such that $\lim_{n \to \infty} x_n = x_0$. Then by definition of the limit 
                we get that 
                $$f'(x_0) = \lim_{n \to \infty} \frac{f(x_n) - f(x_0)}{x_n - x_0} \displayand g'(x_0) = \lim_{n \to \infty} \frac{g(x_n) - g(x_0)}{x_n - x_0} $$
                Now consider 
                \begin{align*}
                    \lim_{n \to \infty} \frac{(fg)(x_n) - (fg)(x_0)}{x - x_0} 
                    &= \lim_{n \to \infty} \frac{f(x_n)g(x_n) - f(x_0)g(x_0)}{x-x_0}  \\
                    &= \lim_{n \to \infty} \frac{f(x_n)g(x_n) - f(x_n)g(x_0) + f(x_n)g(x_0) -f(x_0)g(x_0)}{x-x_0} \\
                    &= \lim_{n \to \infty} \frac{f(x_n)[g(x_n) - g(x_0)] + g(x_0)[f(x_n) - f(x_0)]}{x-x_0} \\
                    &=  \lim_{n \to \infty} \frac{f(x_n)[g(x_n) - g(x_0)]}{x-x_0} + \lim_{n \to \infty} \frac{g(x_0)[f(x_n) - f(x_0)]}{x-x_0} \\
                    &= \lnti f(x_n) \cdot \lnti \frac{g(x_n) - g(x_0)}{x-x_0} + \lnti g(x_0) \cdot \lnti \frac{f(x_n) - f(x_0)}{x-x_0} \\
                    &= f(x_0)g'(x_0) + g(x_0)f'(x_0)
                \end{align*}
                Thus the by definition of the derivative $(fg)'(x_0) = f'(x_0)g(x_0) + g'
                (x_0)f(x_0)$
            \end{solution}
            \break
        \item 
            \begin{enumerate}
                \item Prove that the function $f: \R \setminus \{\frac{1}{2}\} \to \R$ given by $f(x) = \frac{3x+4}{2x-1}$ is differentiable at 
                $x_0 = 1$ and evaluate $f'(1)$
                \item Prove that if $g: \R \to \R$ given by $x^{1/3}$ is not differentiable at $x_0 = 0$ 
            \end{enumerate}
                \begin{solution}
                    \begin{enumerate}
                        \item 
                        Let $(x_n)$ be a sequence in $\R \setminus \{\frac{1}{2}\}$ such that $\lnti x_n = x_0$. Then 
                        \begin{align*}
                            f'(x_0) &= \lnti \frac{f(x_n) - f(x_0)}{x_n - x_0} \\ 
                            &= \lnti \frac{\frac{3x_n + 4}{2x_n -1} -\frac{3x_0 + 4}{2x_0-1}}{x_n-x_0} \\
                            &= \lnti \frac{11(x_0 - x_n)}{(2x_n-1)(2x_0-1)(x_n-x_0)} \\
                            &= \lnti - \frac{11}{(2x_n-1)(2x_0-1)} \\
                            &= \frac{\lnti -11}{\lnti (2x_n - 1)(2x_0 - 1)} \\
                            &= \frac{\lnti -11}{\lnti (2x_n - 1) \cdot \lnti (2x_0 - 1)} \\
                            &= \frac{-11}{(2x_0 - 1)^2}
                        \end{align*}
                        Thus $f$ is differentiable at $x_0 = 1$.
                        $$f'(1) = \frac{-11}{(2(1) - 1)^2} = -11$$
                        \item In order to prove $f$ is not differentiable at 0, we must show some sequence in $\R \setminus \{0\}$ such that it 
                        converges to 0 and $\lnti \frac{f(x_n) - f(0)}{x_n - 0}$  does not exist. Let $x_n = 1/n$. We proved in class that 
                        $\lnti x_n = 0$. Consider
                        \begin{align*}
                            \lnti \frac{f(x_n) - f(0)}{x_n - 0} &= \lnti \frac{f(x_n)}{x_n} \\
                            &= \lnti \frac{(x_n)^{\frac{1}3{}}}{x_n} \\
                            &= \lnti (x_n)^{-\frac{2}{3}} \\
                            &= \lnti \left(\frac{1}{n}\right)^{-\frac{2}{3}} \\
                            &= \lnti n^{2/3}
                        \end{align*}
                        We proved in HW that $\lim_{n \to \infty} \frac{1}{n^p} = 0$ if $p > 0$. Thus $\lnti \frac{1}{n^{\frac{2}{3}}} = 0$. We 
                        also proved in HW that if a sequence $(x_n)$ converges to 0 and $x_n > 0$ for all $n$, then $\frac{1}{x_n}$ diverges to 
                        infinity. $\frac{1}{n^{\frac{2}{3}}} > 0$ for all $n$ and converges to $0$. Thus $\lnti n^{\frac{2}{3}}$ diverges to 
                        infinity. Thus since the limit diverges, $f$ is not differentiable at 0.
                    \end{enumerate}
                \end{solution}
            \break
        \item Prove that if $f: D \to \R$ is differentiable at a point $a \in D$ then $f$ is continious at $a$
            \begin{solution}
                Assume $f$ is differentiable. Let $(x_n)$ be a sequence in $D \setminus \{a\}$ such 
                that $\lnti x_n = a$. Then $\lnti \frac{f(x_n) - f(a)}{x_n - a}$ exists and is the 
                value of the derivative of $f$ at $a$. Now consider
                \begin{align*}
                    \lnti f(x_n) &= \lnti f(x_n) - f(a) + f(a) \\
                                &= \lnti f(x_n) - f(a) + \lnti f(a) \\
                                &= \lnti \frac{f(x_n) - f(a)}{x_n - a} \cdot (x_n -a) + f(a) \\
                                &= \lnti \frac{f(x_n) - f(a)}{x_n - a} \cdot \lnti (x_n -a) +  f(a) \\
                                &= f'(a) \cdot (a-a) + f(a) \\
                                &= 0 + f(a) \\
                                &= f(a)
                \end{align*}
                We have shown that for any sequence $(x_n)$ in $D \setminus \{a\}$ that converges 
                to $a$, the sequence $(f(x_n))$ converges to $f(a)$, thus $f$ is continous at $a$.
            \end{solution}
        \item 
            \begin{enumerate}
                \item Suppose $f: D \to \R$ is a differentiable function, that $D$ contains an open interval $(a, b)$ for some $a < b$, and that 
                $f'(x) > 0$ for all $x \in (a, b)$. Prove that $f$ is strictly increasing on $(a, b)$. That is, prove that if $a < x < y < b$, 
                then $f(x) < f(y)$.
                \item Suppose $f: D \to \R$ is a differentiable function, that $D$ contains an open interval $(a, b)$ for some $a < b$, and that 
                $f'(x) < 0$ for all $x \in (a, b)$. Prove that $f$ is strictly decreasing on $(a, b)$. 
            \end{enumerate}
                \begin{solution}
                    \begin{enumerate}
                        \item Let $x, y \in (a,b)$ such that $x < y$. Note that $[x, y] \subseteq (a, b)$. Thus $f$ is differentiable over $(x, 
                        y)$ and continious over $[x, y]$. Thus we can apply we Mean Value Theorem, so there exists some $x_0 \in (x, y)$ such 
                        that 
                        $$f'(x_0) = \frac{f(y) - f(x)}{y - x}$$ rearranging we get that $$f'(x_0) \cdot (y - x) = f(y) - f(x)$$ note that $y - x 
                        > 0$ and $f'(x_0) > 0$ by assumption. Thus $f(y) - f(x) > 0$ implying that $f(y) > f(x)$. 
                        Hence $f$ is strictly increasing on $(a, b)$
                        \item Let $x, y \in (a, b)$ such that $x < y$. Note that $[x, y] \subseteq (a, b)$. Thus $f$ is differentiable over $(x, 
                        y)$ and continious over $[x, y]$. Thus we can apply we Mean Value Theorem, so there exists some $x_0 \in (x, y)$ such 
                        that $$f'(x_0) = \frac{f(y) - f(x)}{y - x}$$ rearranging we get that $$f'(x_0) \cdot (y - x) = f(y) - f(x)$$ note that 
                        $f'(x_0) < 0$ and $y - x > 0$ thus their product $f'(x_0) \cdot (y - x) < 0$, which gives us $f(y) - f(x) < 0$ implying that $f(y) < f(x)$. Hence $f$ is strictly decreasing on $(a, b)$
                    \end{enumerate}
                \end{solution}
    \end{enumerate}


\end{document}