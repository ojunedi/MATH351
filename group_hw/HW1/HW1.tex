\documentclass[11pt]{exam}
\usepackage[margin=1in]{geometry}
\usepackage{amsfonts, amsmath, amssymb, amsthm}
\usepackage{mathtools}
\usepackage{enumerate}
\usepackage{listings}
\usepackage{array}

\usepackage{fullpage} % margin formatting
\usepackage{enumitem} % configure enumerate and itemize
\usepackage{amsmath, amsfonts, amssymb} % math symbols
\usepackage{xcolor, colortbl} % colors, including in tables
\usepackage{makecell} % thicker \Xhline in table
\usepackage{graphicx} % images, resizing
\usepackage[T1]{fontenc}
% sometimes needed packages
% \usepackage{hyperref} % hyperlinks
% \usepackage{logicproof} % natural deduction
\usepackage{tikz} % drawing graphs
\usepackage{algpseudocode} % pseudocode

% paragraph formatting
\setlength{\parskip}{6pt}
\setlength{\parindent}{0cm}

% newline after Solution:
\renewcommand{\solutiontitle}{\noindent\textbf{Solution:}\par\noindent}

% less space before itemize/enumerate
\setlist{topsep=0pt}

% creates \filcl to grey out cells for groupwork grading
\newcommand{\filcl}{\cellcolor{gray!25}}


% creates \probnum to get the problem number
\newcounter{probnumcount}
\setcounter{probnumcount}{1}
\newcommand{\probnum}{\arabic{probnumcount}. \addtocounter{probnumcount}{1}}

% use roman numerals by default
% \setlist[enumerate]{label={(\roman*)}}

% creates custom list environments for grading guidelines, question parts
\newlist{guidelines}{itemize}{1}
\setlist[guidelines]{label={}, left=0pt .. \parindent, nosep}
\newlist{gwguidelines}{enumerate}{1}
\setlist[gwguidelines]{label={(\roman*)}, nosep}
\newlist{qparts}{enumerate}{2}
\setlist[qparts]{label={(\alph*)}}
\newlist{qsubparts}{enumerate}{2}
\setlist[qsubparts]{label={(\roman*)}}
\newlist{stmts}{enumerate}{1}
\setlist[stmts]{label={(\roman*)}, nosep}



% in order to compile this file you need to get 'header.tex' from
% Canvas and change the line below to the appropriate file path
\input{"/Users/omerjunedi/Documents/math351/group_hw/header.tex"}

% TODO: always change
\newcommand{\hwnum}{1}
\newcommand{\duedate}{Jan 29}
\usepackage{xcolor}

\hwheader  % header for homework
%\hwslnheader   % header for homework solutions

%Comment out this line to hid "Solution: ..." boxes.
\printanswers

\begin{document}
    \begin{enumerate}
        \item Prove that the sequence $(-1)^n$ does not converge.
            \begin{solution}
                Seeking contradiction assume that $(-1)^n$ converges to some number $a$ $$\lim_{n \to \infty} (-1)^n = a$$

                Fix $\e = 1/2$ then by definition of convergence there is some $N \in \R$ such that for all $n  > N$  $$\left|(-1)^n - a\right| < 1/2$$
                \underline{For odd $n > N$}
                \begin{align*}
                    \left|-1 - a\right| < 1/2 \\
                    -1/2 < -1 - a < 1/2 \\
                    1/2 < -a < 3/2 \\
                    -3/2 < a < -1/2
                \end{align*}
                \underline{For even $n > N$}
                \begin{align*}
                    \left|1 - a\right| < 1/2 \\
                    -1/2 < 1 - a < 1/2 \\
                    -3/2 < -a < -1/2 \\
                    1/2 < a < 3/2
                \end{align*}
            We have shown $a \in (-3/2, 1/2)$ and that $a \in (1/2, 3/2)$ which is a contradiction. Thus $(-1)^n$ does not converge.
            \end{solution}
        \item Prove that every Cauchy sequence is bounded.
            \begin{solution}
                Assume that $(x_n)$ is a Cauchy sequence then for every $\e > 0$ there exists some $N \in \R$ such that for all $n, m > N$ we have $|x_n - x_m| < \e$. Note that $|x_n| = |x_n - x_m + x_m| \leq |x_n - x_m| + |x_m|$. Fix $\e = 1$, combining this with the Cauchy criterion we get that for $n, m > N$
                $$|x_n| \leq |x_n - x_m| + |x_m| < |x_m| + 1$$
                Let $m = N + 1$ then we get that 
                $$|x_n| < |x_{N+1}| + 1$$
                this is true for all $n > N$. This bounds all the terms past the Nth term. For all terms before the Nth terms we can bound it by the maximum of the terms thus for $n \leq N$
                $$ |x_n| \leq \text{max}\{|x_1|, |x_2|, \cdots , |x_N|\}$$
                Thus to bound all terms we can choose the maximum of $|x_{N+1}| + 1$ and $\text{max}\{|x_1|, |x_2|, \cdots , |x_N|\}$. Let $R = \text{max}\{|x_1|, |x_2|, \cdots , |x_N|, 1 + |x_{N+1}|\}$. Then $|x_n| \leq R$ for all $n \in \N$ and thus $(x_n)$ is bounded. 
            \end{solution}
        \item Prove that if $(x_n)$ and $(y_n)$ are convergent sequenecs, $\lim_{n \to \infty} x_n = x$ and $\lim_{n \to \infty} y_n = y$ and $x_n \leq y_n$ for all $n \in \N$, then $x \leq y$
            \begin{solution}
                We have that $y_n - x_n \geq 0$ and by limit laws we have that $(y_n - x_n)$ converges to $y-x$. If we can show that $y - x \geq 0$ then we have the desired result. By problem 10 on the worksheet we showed that if $(s_n)$ is a convergent sequence and $s_n \geq 0$ for all but finitely many values of $n$, then $\lim_{n \to \infty} s_n \geq 0$. Applying this result to $(y_n - x_n)$ we get that $y - x \geq 0$ and thus $x \leq y$. 
            \end{solution}
        \item Suppose that $(a_n), (b_n)\text { and } (s_n)$ are three sequenecs and that
        $$a_n \leq s_n \leq b_n$$
        for all $n \in \N$. Prove that if $(a_n)$ and $(b_n)$ both converge to $s$, then $(s_n)$ also converges to $s$. 
            \begin{solution}
                Fix $\e > 0$. $(a_n)$ converges to $s$ thus there exists $N_1 \in \R$ such that 
                $|a_n - s| < \e$ likewise $(b_n)$ converges to $s$ thus there there exists $N_2 \in \R$ such that $|b_n - s| < \e$. Let $N = \text{max}\{N_1, N_2\}$ then  by combining $a_n \leq s_n \leq b_n$ and $a_n$ converging to $s$ $$-\e < a_n - s < \e$$ and that $b_n$ converges to $s$ 
                $$-\e < b_n - s < \e$$ we have that for $n > N$
                $$s_n -s \leq b_n -s < \e$$ and 
                $$-\e < a_n -s \leq s_n -s$$ 
                thus $|s_n - s| < \e$. Thus for all $\e > 0$ we have shown an $N \in \R$ such that if $n > N$ then $|s_n - s| < \e$, thus $(s_n)$ converges to $s$.
            \end{solution}
    \end{enumerate}
\end{document}