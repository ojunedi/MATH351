\documentclass[11pt]{exam}
\usepackage[margin=1in]{geometry}
\usepackage{amsfonts, amsmath, amssymb, amsthm}
\usepackage{mathtools}
\usepackage{enumerate}
\usepackage{listings}
\usepackage{array}

\usepackage{fullpage} % margin formatting
\usepackage{enumitem} % configure enumerate and itemize
\usepackage{amsmath, amsfonts, amssymb} % math symbols
\usepackage{xcolor, colortbl} % colors, including in tables
\usepackage{makecell} % thicker \Xhline in table
\usepackage{graphicx} % images, resizing
\usepackage[T1]{fontenc}
% sometimes needed packages
% \usepackage{hyperref} % hyperlinks
% \usepackage{logicproof} % natural deduction
\usepackage{tikz} % drawing graphs
\usepackage{algpseudocode} % pseudocode

% paragraph formatting
\setlength{\parskip}{6pt}
\setlength{\parindent}{0cm}

% newline after Solution:
\renewcommand{\solutiontitle}{\noindent\textbf{Solution:}\par\noindent}

% less space before itemize/enumerate
\setlist{topsep=0pt}

% creates \filcl to grey out cells for groupwork grading
\newcommand{\filcl}{\cellcolor{gray!25}}


% creates \probnum to get the problem number
\newcounter{probnumcount}
\setcounter{probnumcount}{1}
\newcommand{\probnum}{\arabic{probnumcount}. \addtocounter{probnumcount}{1}}

% use roman numerals by default
% \setlist[enumerate]{label={(\roman*)}}

% creates custom list environments for grading guidelines, question parts
\newlist{guidelines}{itemize}{1}
\setlist[guidelines]{label={}, left=0pt .. \parindent, nosep}
\newlist{gwguidelines}{enumerate}{1}
\setlist[gwguidelines]{label={(\roman*)}, nosep}
\newlist{qparts}{enumerate}{2}
\setlist[qparts]{label={(\alph*)}}
\newlist{qsubparts}{enumerate}{2}
\setlist[qsubparts]{label={(\roman*)}}
\newlist{stmts}{enumerate}{1}
\setlist[stmts]{label={(\roman*)}, nosep}



% in order to compile this file you need to get 'header.tex' from
% Canvas and change the line below to the appropriate file path
\input{"/Users/omerjunedi/Documents/math351/individual_hw/header.tex"}

% TODO: always change
\newcommand{\hwnum}{4/3}
\newcommand{\duedate}{April 3}
\usepackage{xcolor}

\hwheader  % header for homework
%\hwslnheader   % header for homework solutions

%Comment out this line to hid "Solution: ..." boxes.
\printanswers





\begin{document}



    \begin{enumerate}
        

        \item 
        \begin{enumerate}
            \item Prove that a constant function $f(x) = c$ for some $c\in\R$ is differentiable at all $a \in \R$
            and $f'(a) = 0$
            \item Prove that a linear function \(f(x) = mx + b\), where \(m, b \in \mathbb{R}\), is differentiable at all \(a \in \mathbb{R}\) and \(f'(a) = m\).
        \end{enumerate}
            \begin{solution}
                \begin{enumerate}
                    \item Fix some $a \in \R$. Let $(x_n)$ be a sequence in $\R \setminus \{a\}$ such that $\lnti x_n = a$, then 
                    $$f'(a) = \lnti \frac{f(x_n) - f(a)}{x_n - a} = \lnti \frac{c - c}{x_n - x} = 0$$
                    The limit exists, thus $f$ is differentiable for all $a \in \R$ with $f'(a) = 0$
                    \item Let $a \in \R$.  Let $(x_n)$ be a sequence in $\R \setminus \{a\}$ such that $\lnti x_n = a$, then 
                    $$f'(a) = \lnti \frac{f(x_n) - f(a)}{x_n - a} = \lnti \frac{mx_n + b - (ma + b)}{x_n - a} = \lnti \frac{m(x_n-a)}{x_n -a} = m$$
                    The limit exists, thus $f$ is differentiable for all $a \in \R$ with $f'(a) = m$
                \end{enumerate}
            \end{solution}

        \item Suppose \(g: D \rightarrow \mathbb{R}\) is differentiable at \(a\) and that \(g(a) \neq 0\). Prove that the function \( \frac{1}{g}(x) = \frac{1}{g(x)} \) is differentiable at \(a\) and that 
        \[
        \left( \frac{1}{g} \right)'(a) = -\frac{g'(a)}{g^2(a)}.
        \]
        (Notice that it follows from the fact that \(g\) is continuous at \(a\) with \(g(a) \neq 0\) that \(g\) is non-zero on some open interval about \(a\), so that \(\frac{1}{g}\) is defined on an open interval about \(a\).)
            \begin{solution}
                Assume $g: D \to \R$ is differentiable at $a$ with $g(a) \neq 0$. Let $(x_n)$ be a sequence in $D \setminus \{a\}$ such that $\lnti x_n = a$. Then 
                \begin{align}
                    \left( \frac{1}{g} \right)'(a) &= \lnti \frac{\frac{1}{g(x_n)} - \frac{1}{g(a)}}{x_n - a} \\ &= \lnti \frac{g(a) - g(x_n)}{g(a)\cdot g(x_n)\cdot(x_n-a)} \\
                        &= \lnti \frac{g(a) - g(x_n)}{x_n - a} \cdot \lnti \frac{1}{g(a)\cdot g(x_n)} \\
                        &= -g'(a) \cdot \lnti \frac{1}{g(a)} \cdot \lnti \frac{1}{g(x_n)} \\
                        &= -g'(a) \cdot \frac{1}{g(a)} \cdot \frac{1}{g(a)} \\
                        &= -\frac{g'(a)}{g^2(a)}
                \end{align}
                Note since $g$ is differentiable at $a$, it is continuous at $a$. Thus $\lnti g(x_n) = g(a)$ and 
                it follows from there that $\lnti \frac{1}{g(x_n)} = \frac{1}{g(a)}$
            \end{solution}
        \item Suppose \(f: \mathbb{R} \rightarrow \mathbb{R}\) is defined as \(f(x) = x^2\) for \(x \geq 0\) and \(f(x) = x\) for \(x < 0\). Prove that \(f\) is not differentiable at \(0\).
            \begin{solution}
                To show a function is not differentiable a point $a$ we need to show the limit for the definitino 
                of the derivative does not exist. We can do this by showing  two sequences that converge to $a$ 
                such that their limits under the defintion of the derivative are not equal. Let $x_n = 1/n$ and 
                let $y_n = -1/n$ for all $n$. We proved in class both of these sequnces converge to 0. Note 
                that $x_n > 0$ for all $n$,
                $$f'(0) = \lnti \frac{f(x_n) - f(0)}{x_n - 0} = \lnti \frac{\left(\frac{1}{n}\right)^2}{\frac{1}
                {n}} = \lnti \frac{1}{n} = 0$$ 
                and note that $y_n < 0$ for all $n$ so,
                $$f'(0) = \lnti \frac{f(y_n) - f(0)}{y_n - 0} = \lnti \frac{1/n}{1/n} = \lnti 1 = 1$$
                thus since the limit values do not agree we have shown that the derivative does not exist at $x = 
                0$. 
            \end{solution}
        \item Suppose that \(f: D \rightarrow \mathbb{R}\) and \(g: D \rightarrow \mathbb{R}\) are both uniformly continuous on \(D\). Prove that the function \(f + g: D \rightarrow \mathbb{R}\) is uniformly continuous on \(D\).
            \begin{solution}
                Fix $\e > 0$. Since $f$ is uniformly continuous there exists some $\d_f > 0$ 
                such that  for all $x, y \in D$ if $|x - y| < \d_f$ then $|f(x) - f(y)| < \e/2$. Likewise since $g$ is uniformly 
                continuous there exists some $\d_g > 0$ such that for all $x, y \in D$ if $|x - y| < \d_g$ then $|g(x) - g(y)| < \e/2$. Let $\d = \min\{\d_f, \d_g\}$ then for all $x, y \in D$ if $|x-y| < \d$ we have that 
                \begin{align*}
                    |(f+g)(x) - (f+g)(y)| &= |f(x)-f(y) + g(x) - g(y)| \\
                                        &\leq |f(x) - f(y)| + |g(x) - g(y)| \\
                                        &< \e/2 + \e/2 \\
                                        &= \e
                \end{align*}
                We have shown for all $\e > 0$ there exists some $\d > 0$ such that for all $x, y \in D$ if $|x - y| < \d$ then $|f(x) - f(y)| < \e$. Thus $f+g$ is uniformly continuous on $D$.
            \end{solution}





    \end{enumerate}



\end{document}