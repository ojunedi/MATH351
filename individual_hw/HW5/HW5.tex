\documentclass[11pt]{exam}
\usepackage[margin=1in]{geometry}
\usepackage{amsfonts, amsmath, amssymb, amsthm}
\usepackage{mathtools}
\usepackage{enumerate}
\usepackage{listings}
\usepackage{array}

\usepackage{fullpage} % margin formatting
\usepackage{enumitem} % configure enumerate and itemize
\usepackage{amsmath, amsfonts, amssymb} % math symbols
\usepackage{xcolor, colortbl} % colors, including in tables
\usepackage{makecell} % thicker \Xhline in table
\usepackage{graphicx} % images, resizing
\usepackage[T1]{fontenc}
% sometimes needed packages
% \usepackage{hyperref} % hyperlinks
% \usepackage{logicproof} % natural deduction
\usepackage{tikz} % drawing graphs
\usepackage{algpseudocode} % pseudocode

% paragraph formatting
\setlength{\parskip}{6pt}
\setlength{\parindent}{0cm}

% newline after Solution:
\renewcommand{\solutiontitle}{\noindent\textbf{Solution:}\par\noindent}

% less space before itemize/enumerate
\setlist{topsep=0pt}

% creates \filcl to grey out cells for groupwork grading
\newcommand{\filcl}{\cellcolor{gray!25}}


% creates \probnum to get the problem number
\newcounter{probnumcount}
\setcounter{probnumcount}{1}
\newcommand{\probnum}{\arabic{probnumcount}. \addtocounter{probnumcount}{1}}

% use roman numerals by default
% \setlist[enumerate]{label={(\roman*)}}

% creates custom list environments for grading guidelines, question parts
\newlist{guidelines}{itemize}{1}
\setlist[guidelines]{label={}, left=0pt .. \parindent, nosep}
\newlist{gwguidelines}{enumerate}{1}
\setlist[gwguidelines]{label={(\roman*)}, nosep}
\newlist{qparts}{enumerate}{2}
\setlist[qparts]{label={(\alph*)}}
\newlist{qsubparts}{enumerate}{2}
\setlist[qsubparts]{label={(\roman*)}}
\newlist{stmts}{enumerate}{1}
\setlist[stmts]{label={(\roman*)}, nosep}



% in order to compile this file you need to get 'header.tex' from
% Canvas and change the line below to the appropriate file path
\input{"/Users/omerjunedi/Documents/math351/individual_hw/header.tex"}

% TODO: always change
\newcommand{\hwnum}{5}
\newcommand{\duedate}{Feb 5}
\usepackage{xcolor}

\hwheader  % header for homework
% \hwslnheader   % header for homework solutions

%Comment out this line to hid "Solution: ..." boxes.
\printanswers

\begin{document}
    \begin{enumerate}
        \item Prove or disprove the following claim: \textit{There exists a subset of $\R$ that is closed and not bounded}.
            \begin{solution}
                The statement is true. Consider $\R$, which is a subset of $\R$. Let $(x_n)$ be a convergent sequence that converges to $x$ such that $x_n \in \R$ for all $n$. By definition $x \in \R$, thus $\R$ is closed. Seeking contradiction, assume that $\R$ is bounded and that $\alpha$ is its upper bound. Then clearly $\alpha > 0$, but we also have that $1 + \alpha > \alpha$ where $1 + \alpha \in \R$, but this contradicts that $\alpha$ is an upper bound. Thus it must be that $\R$ is unbounded. Therefore we have shown a subset of $\R$ (namely $\R$) that is closed and unbounded.
            \end{solution}
        \item Prove that if $S$ is a non-empty subset of a bounded set $T$, then $$\inf T \leq \inf S \leq \sup S \leq \sup T$$
            \begin{solution}
                $s \leq \sup S$ for all $s \in S$ and $t \leq \sup T$ for all $t \in T$ by definition of supremum. Since $S \subseteq T$ for all $s \in S$, $s \in T$ hence $s \leq \sup T$ for all $s \in S$. Notice that $\sup T$ is an upper bound for $S$, thus $\sup S \leq \sup T$ since $\sup S$ is the \textit{least} upper bound. $\inf S \leq s$ for all $s \in S$ and $\inf T \leq t$ for all $t \in T$ by definition of infimum. Since $S \subseteq T$ for all $s \in S$, $s \in T$ thus $\inf T \leq s$. Notice that $\inf T$ is a lower bound for $S$, thus $\inf T \leq \inf S$ since $\inf S$ is the \textit{greatest} lower bound. Combining the inequalites we get that 
                $$\inf T \leq \inf S \leq s \leq \sup S \leq \sup T$$
            \end{solution}
        \item Let $S$ be a nonempty subset of $\R$. Prove that $-\sup S = \inf -S$ where $-S = \{-s | s \in S\}$
            \begin{solution}
                For all $s \in S$ we have $$s \leq \sup S \Longrightarrow -s \geq -\sup S$$ Thus $-\sup S$ is a lower bound for $-S$. So we have that $$-\sup S \leq  \inf -S$$ since $\inf - S$ is the greatest lower bound. 
                For all $-s \in -S$ we have that $$\inf -S \leq -s \Longrightarrow -\inf -S \geq s$$ Notice that $-\inf -S$ is an upper bound for $S$. Thus $$\sup S \leq -\inf -S \Longrightarrow -\sup S \geq \inf -S$$ So we have shown that $-\sup S \leq  \inf -S$ and that $-\sup S \geq \inf -S$ thus $$-\sup S = \inf -S$$
            \end{solution}
        \item Use results we have already established to prove that if $(s_n)$ and $(t_n)$ are convergent, $s = \lim s_n$, $t = \lim t_n \neq 0$, and $t_n$ is non-zero for all $n$, then $\left(\frac{s_n}{t_n}\right)$ converges to $\frac{s}{t}$
            \begin{solution}
                $$\lim_{n \to \infty} \frac{s_n}{t_n} = \lim_{n \to \infty} s_n \cdot \frac{1}{t_n} = \lim_{n \to \infty} s_n \cdot \lim_{n \to \infty} \frac{1}{t_n} = s \cdot \frac{1}{t} = \frac{s}{t}$$
                We proved multiplication limit law in class and from previous HW we proved that if $(x_n)$ converges to $x \neq 0$ and $x_n$ is non-zero for all $n$, then $\left(\frac{1}{x_n}\right)$ converges to $\frac{1}{x}$. Which I applied to $\left(\frac{1}{t_n}\right)$
            \end{solution}
    \end{enumerate}
\end{document}