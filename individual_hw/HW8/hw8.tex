\documentclass[11pt]{exam}
\usepackage[margin=1in]{geometry}
\usepackage{amsfonts, amsmath, amssymb, amsthm}
\usepackage{mathtools}
\usepackage{enumerate}
\usepackage{listings}
\usepackage{array}

\usepackage{fullpage} % margin formatting
\usepackage{enumitem} % configure enumerate and itemize
\usepackage{amsmath, amsfonts, amssymb} % math symbols
\usepackage{xcolor, colortbl} % colors, including in tables
\usepackage{makecell} % thicker \Xhline in table
\usepackage{graphicx} % images, resizing
\usepackage[T1]{fontenc}
% sometimes needed packages
% \usepackage{hyperref} % hyperlinks
% \usepackage{logicproof} % natural deduction
\usepackage{tikz} % drawing graphs
\usepackage{algpseudocode} % pseudocode

% paragraph formatting
\setlength{\parskip}{6pt}
\setlength{\parindent}{0cm}

% newline after Solution:
\renewcommand{\solutiontitle}{\noindent\textbf{Solution:}\par\noindent}

% less space before itemize/enumerate
\setlist{topsep=0pt}

% creates \filcl to grey out cells for groupwork grading
\newcommand{\filcl}{\cellcolor{gray!25}}
\newcommand{\qs}{\sum_{n=m}^{\infty}}

% creates \probnum to get the problem number
\newcounter{probnumcount}
\setcounter{probnumcount}{1}
\newcommand{\probnum}{\arabic{probnumcount}. \addtocounter{probnumcount}{1}}

% use roman numerals by default
% \setlist[enumerate]{label={(\roman*)}}

% creates custom list environments for grading guidelines, question parts
\newlist{guidelines}{itemize}{1}
\setlist[guidelines]{label={}, left=0pt .. \parindent, nosep}
\newlist{gwguidelines}{enumerate}{1}
\setlist[gwguidelines]{label={(\roman*)}, nosep}
\newlist{qparts}{enumerate}{2}
\setlist[qparts]{label={(\alph*)}}
\newlist{qsubparts}{enumerate}{2}
\setlist[qsubparts]{label={(\roman*)}}
\newlist{stmts}{enumerate}{1}
\setlist[stmts]{label={(\roman*)}, nosep}



% in order to compile this file you need to get 'header.tex' from
% Canvas and change the line below to the appropriate file path
\input{"/Users/omerjunedi/Documents/math351/individual_hw/header.tex"}

% TODO: always change
\newcommand{\hwnum}{3/20}
\newcommand{\duedate}{March 20}
\usepackage{xcolor}

\hwheader  % header for homework
%\hwslnheader   % header for homework solutions

%Comment out this line to hid "Solution: ..." boxes.
\printanswers


\begin{document}
    \begin{enumerate}
        \item Prove using the $\e$-$\d$ definition of continuity, that $h(x) = 2x^3$ is continious at 1.
            \begin{solution}
                Notice that  if $|x-1|<\d$ then 
                $1-\d < x < 1 + \d$ from which we get 
                $$(1-\d)^2 < x^2  < (1+\d)^2$$ and $$2-\d < x+1 < 2 + \d$$ adding both of these inequalities we get that 
                $$(2-\d)+(1-\d)^2 < x^2 + x + 1 < (1+\d)^2 + 2 + \d$$
                then if $\d < 1$
                $$x^2 + x + 1 < (1+1)^2 + 2 + 1 = 7$$

                \begin{proof}
                    Fix $\e > 0$. Let $\d = \min\{1, \frac{\e}{14}\}$. Then if $|x-1| < \d$ we have
                    \begin{align*}
                        |f(x)-f(1)| &= |2x^3-2|\\ &= 2|x^3-1|\\ &= 2|x-1||x^2+x+1| \\
                                                          &< 2 \cdot \frac{\e}{14} \cdot 7 \\
                                                          &= \e
                    \end{align*}
                \end{proof}

            \end{solution}
        \item Suppose that $f: D \to \R$ is continious at $x_0 \in D$ in the $\e$-$\d$ definition of 
            continuity. Prove that if $(x_n)$ is a sequence in D converginh to $x_0$, then $(f(x_n))$ 
            converges to $f(x_0)$.
            \begin{solution}
                $f: D \to \R$ is continious at $x_0 \in D$ then by definition we know for all $\e > 0$ 
                there exists some $\d > 0$ such that if $|x-x_0| < \d$ then $|f(x) -f(x_0)| < \e$. Now 
                let $(x_n)$ be a sequence in $D$ that converges to $x_0 \in D$. Then by definition of 
                sequence convergence we know for all $\e > 0$ there exists some $N \in \R$ such that 
                $|x_n - x_0| < \e$ for all $n > N$. We want to show show that $(f(x_n))$ converges to $f(x_0)$ or in other words that for all $\e > 0$ there exists some $N \in \R$ such that 
                $|f(x_n) - f(x_0)| < \e$ for all $n > N$.
                \begin{proof}
                    Fix $\e > 0$. Assume $f: D \to \R$ is continious at $x_0$ then we know there exists 
                    some $\d > 0$ such that if $|x-x_0| < \d$ then $|f(x)-f(x_0)| < \e$. Now since 
                    $(x_n)$ converges to $x_0$ we can find some $N \in \R$ such that for $n > N$ the 
                    distance between $x_n$ and $x_0$ is any value we want. Let this distance be the 
                    aforementioned $\d$ (we are setting the $\e$ in the definition for sequence 
                    convergence to $\d$). So for some $N_f \in \R$ if $n > N_f$ then $|x_n - x_0| < \d$
                    and by assumption that $f$ is continous this implies that $|f(x_n)-f(x_0)| < \e$. Thus $f(x_n)$ converges to $f(x_0)$.
                \end{proof}
            \end{solution}
        \item Prove that if a function $f: D \to \R$ does not satisfy the $\e$-$\d$ definition of 
        continuity at some $x_0 \in D$, then it does not satisfy the sequence definition of continuity
        at $x_0$
            \begin{solution}
                skip
            \end{solution}
        \item Prove that if $\qs a_n = s$ for some real number $s$, then $\lim_{n \to \infty} a_n = 0$ 
            \begin{solution}
                Assume $\qs a_n = s$ for some real number $s$. Then by definition the sequence of the 
                partial sums $(s_n)_{n=m}^\infty$ converges to $s$. Notice that 
                $$s_k = \sum_{n=m}^{k} a_n = \sum_{n=m}^{k-1} a_n + a_k$$ and that 
                $$s_{k-1} = \sum_{n=m}^{k-1} a_n$$ thus combining the two equations we get 
                $$s_k = s_{k-1} + a_k \Longrightarrow a_k = s_k - s_{k-1}$$
                Thus 
                $$\lim_{k \to \infty} a_k = \lim_{k \to \infty} s_k - s_{k-1} = \lim_{k \to \infty}
                s_k - \lim_{k \to \infty} s_{k-1} = s - s = 0
                $$
            \end{solution}
    \end{enumerate}
\end{document}