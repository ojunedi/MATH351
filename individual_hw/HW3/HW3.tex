\documentclass[11pt]{exam}
\usepackage[margin=1in]{geometry}
\usepackage{amsfonts, amsmath, amssymb, amsthm}
\usepackage{mathtools}
\usepackage{enumerate}
\usepackage{listings}
\usepackage{array}

\usepackage{fullpage} % margin formatting
\usepackage{enumitem} % configure enumerate and itemize
\usepackage{amsmath, amsfonts, amssymb} % math symbols
\usepackage{xcolor, colortbl} % colors, including in tables
\usepackage{makecell} % thicker \Xhline in table
\usepackage{graphicx} % images, resizing
\usepackage[T1]{fontenc}
% sometimes needed packages
% \usepackage{hyperref} % hyperlinks
% \usepackage{logicproof} % natural deduction
\usepackage{tikz} % drawing graphs
\usepackage{algpseudocode} % pseudocode

% paragraph formatting
\setlength{\parskip}{6pt}
\setlength{\parindent}{0cm}

% newline after Solution:
\renewcommand{\solutiontitle}{\noindent\textbf{Solution:}\par\noindent}

% less space before itemize/enumerate
\setlist{topsep=0pt}

% creates \filcl to grey out cells for groupwork grading
\newcommand{\filcl}{\cellcolor{gray!25}}


% creates \probnum to get the problem number
\newcounter{probnumcount}
\setcounter{probnumcount}{1}
\newcommand{\probnum}{\arabic{probnumcount}. \addtocounter{probnumcount}{1}}

% use roman numerals by default
% \setlist[enumerate]{label={(\roman*)}}

% creates custom list environments for grading guidelines, question parts
\newlist{guidelines}{itemize}{1}
\setlist[guidelines]{label={}, left=0pt .. \parindent, nosep}
\newlist{gwguidelines}{enumerate}{1}
\setlist[gwguidelines]{label={(\roman*)}, nosep}
\newlist{qparts}{enumerate}{2}
\setlist[qparts]{label={(\alph*)}}
\newlist{qsubparts}{enumerate}{2}
\setlist[qsubparts]{label={(\roman*)}}
\newlist{stmts}{enumerate}{1}
\setlist[stmts]{label={(\roman*)}, nosep}



% in order to compile this file you need to get 'header.tex' from
% Canvas and change the line below to the appropriate file path
\input{"/Users/omerjunedi/Documents/math351/individual_hw/header.tex"}

%TODO: change
\newcommand{\hwnum}{3}
\newcommand{\duedate}{Jan 24}
\usepackage{xcolor}


\hwheader  % header for homework
%\hwslnheader   % header for homework solutions

%Comment out this line to hid "Solution: ..." boxes.
\printanswers

\begin{document}
    \begin{enumerate}
        \item Prove that the sequence $(\sin(\frac{n\pi}{3}))$ does not converge
            \begin{solution}
                skip
            \end{solution}

        \item Prove that if $(a_n)$ and $(b_n)$ are two convergent sequences such that $\lim_{n \to \infty} a_n = a$ and $\lim_{n \to \infty} b_n = b$, then $(a_nb_n)$ converges to $ab$. (This is the Multiplication Limit Law for
        sequences.)
            \begin{solution}
                To prove the Multiplication Limit Law for
                sequences, I will first prove that convergent sequences are bounded. 
                \begin{proof}
                    Assume that $(a_n)$ is a converges to $a$. By the definition of convergence (with $\varepsilon = 1$) there is some $N$ where 
                    $$|a_n - a| < 1$$
                    for all $n > N$. That is, $a-1 < a_n < a + 1$ for all $n > N$. Let 
                    $$ U = \text{max}\{a_1, a_2, \ldots , a_N, a + 1\}$$ and 
                    $$ L = \text{min}\{a_1, a_2, \ldots , a_N, a - 1\}$$

                    Note that if $n \leq N$, then $L \leq a_n \leq U$, since each $a_n$ is included in the sets we are taking the minimum and maximum of. And if $n > N$, then we already have shown that $a-1 < a_n < a + 1$, which implies that 
                    $$L \leq a - 1 < a_n < a + 1 \leq U$$
                    and hence we have that 
                    $$L \leq a_n \leq U$$
                    for all n. And thus by definition, $(a_n)$ is bounded. 
                \end{proof}
                Fix $\varepsilon > 0$. Since $(a_n)$ converges we know by above that it is bounded, and by the \textbf{Conseqeunce of Class Exercise} from worksheet 1.4 we know there exists some $R \geq 0$ such that $|b_n| \leq R$ for all $n \in \N$. Let $\varepsilon_1 = \frac{\varepsilon}{2|b| + 1}$ and $\varepsilon_2 = \frac{\varepsilon}{2R + 1}$. Since $\varepsilon_1 > 0$ we know there exists some $N_1$ such that $|a_n - a| < \varepsilon_1$. Likewise, since $\varepsilon_2 > 0$ we know there exists $N_2$ such that $|b_n - b| < \varepsilon_2$. Let $N = \text{max}\{N_1, N_2\}$. Then for any $n > N$, 
                \begin{align*}
                    |a_nb_n - ab| &=     |a_nb_n - a_nb + a_nb - ab| \\
                                  &=     |a_n(b_n - b) + b(a_n - a)| \\
                                  &\leq  |a_n(b_n -b)| + |b(a_n - a)| \\
                                  &=     |a_n| \cdot |b_n - b| + |b| \cdot |a_n - a| \\
                                  &<     |a_n| \cdot \varepsilon_2 + |b| \cdot \varepsilon_1 \\
                                  &=     |a_n| \cdot \frac{\varepsilon}{2R + 1} + |b| \cdot \frac{\varepsilon}{2|b| + 1} \\ 
                                  &<     \frac{\varepsilon}{2} + \frac{\varepsilon}{2} \\
                                  &=     \varepsilon
                \end{align*}
                That is, for $n > N$ we have $|a_nb_n - ab| < \varepsilon$. Therefore $(a_nb_n)$ converges to $ab$. $\hspace*{0em plus 1fill} \qed$
            \end{solution}
        
        \item Prove that every convergent sequence is Cauchy.
        \begin{solution}
            Asssume that $(x_n)$ converges to $x$. Fix $\e > 0$. Notice that $\e/2 > 0$, there exists some $N \in \R$ such that for every $n > N$ we have 
            $$|x_n - x | < \e / 2$$
            Then, for any $n$, $m$ > $N$, 
            \begin{align*}
                |x_n - x_m| &= |x_n - a + a - x_m| \\
                            &\leq |x_n - a| + |x_m - a| \\
                            &< \e/2 + \e/2 \\
                            &= \e
            \end{align*}
            That is, for all $\e > 0$ there exists an $N \in \R$ such that for any $n, m > N$ we have $|x_n - x_m| < \e$, thus $(x_n)$ is Cauchy. $\hspace*{0em plus 1fill} \qed$
        \end{solution}
    \end{enumerate}
\end{document}



% Prove that if (an) and (bn) are two convergent sequences such that lim
% n→∞ an = a and
% lim
% n→∞ bn = b, then (anbn) converges to ab. (This is the Multiplication Limit Law for
% sequences.