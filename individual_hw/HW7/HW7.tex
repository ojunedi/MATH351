\documentclass[11pt]{exam}
\usepackage[margin=1in]{geometry}
\usepackage{amsfonts, amsmath, amssymb, amsthm}
\usepackage{mathtools}
\usepackage{enumerate}
\usepackage{listings}
\usepackage{array}
\usepackage{fullpage} % margin formatting
\usepackage{enumitem} % configure enumerate and itemize
\usepackage{amsmath, amsfonts, amssymb} % math symbols
\usepackage{xcolor, colortbl} % colors, including in tables
\usepackage{makecell} % thicker \Xhline in table
\usepackage{graphicx} % images, resizing
\usepackage[T1]{fontenc}
% sometimes needed packages
% \usepackage{hyperref} % hyperlinks
% \usepackage{logicproof} % natural deduction
\usepackage{tikz} % drawing graphs
\usepackage{algpseudocode} % pseudocode

% paragraph formatting
\setlength{\parskip}{6pt}
\setlength{\parindent}{0cm}

% newline after Solution:
\renewcommand{\solutiontitle}{\noindent\textbf{Solution:}\par\noindent}

% less space before itemize/enumerate
\setlist{topsep=0pt}

% creates \filcl to grey out cells for groupwork grading
\newcommand{\filcl}{\cellcolor{gray!25}}


% creates \probnum to get the problem number
\newcounter{probnumcount}
\setcounter{probnumcount}{1}
\newcommand{\probnum}{\arabic{probnumcount}. \addtocounter{probnumcount}{1}}

% use roman numerals by default
% \setlist[enumerate]{label={(\roman*)}}

% creates custom list environments for grading guidelines, question parts
\newlist{guidelines}{itemize}{1}
\setlist[guidelines]{label={}, left=0pt .. \parindent, nosep}
\newlist{gwguidelines}{enumerate}{1}
\setlist[gwguidelines]{label={(\roman*)}, nosep}
\newlist{qparts}{enumerate}{2}
\setlist[qparts]{label={(\alph*)}}
\newlist{qsubparts}{enumerate}{2}
\setlist[qsubparts]{label={(\roman*)}}
\newlist{stmts}{enumerate}{1}
\setlist[stmts]{label={(\roman*)}, nosep}



% in order to compile this file you need to get 'header.tex' from
% Canvas and change the line below to the appropriate file path
\input{"/Users/omerjunedi/Documents/math351/individual_hw/header.tex"}

% TODO: always change
\newcommand{\hwnum}{3/4}
\newcommand{\duedate}{March 4}
\usepackage{xcolor}

\hwheader  % header for homework
%\hwslnheader   % header for homework solutions

%Comment out this line to hid "Solution: ..." boxes.
\printanswers%                                                                  


\begin{document}

    \begin{enumerate}
        \item Let $f: \R \to \R$ be defined by $f(x) = \sin\left(\frac{1}{x}\right)$ 
        if $x \neq 0$ and $f(0) = 0$. Prove that $f$ is not continious at 0.
            \begin{solution}
                Let $x_n = \frac{1}{n}$ for all $n \in \N$. Note that $\frac{1}{n} > 0$ for all $n$. Thus $f(x_n) = \sin\left(\frac{1}{x_n}\right) = \sin(n)$. We proved in class that $\lim_{n \to \infty} \frac{1}{n} = 0$. Now consider $$\lim_{n \to \infty} f(x_n) = \lim_{n \to \infty} \sin(n)$$ which we proved in class diverges. Thus we have that $(x_n)$ converges to $0$ but $(f(x_n))$ does not converge to $f(0)$, thus f is not continious at $x = 0$.
            \end{solution}
        \item Suppose that $f: D \to \R$, $g: D \to \R$ and $h: D \to \R$ are 3 functions and that $$f(z) \leq g(z) \leq h(z)$$ for all $z \in D$. Show that if $f$ and $h$ are both continious at $x \in D$ and $f(x) = h(x)$, then $g$ is also continious at $x$.
            \begin{solution}
                Let $(x_n)$ be a sequence in $D$ that converges to $x$. By continuity of $f$ and $h$ and that $f(x) = h(x)$ we know $$\lim_{n \to \infty} f(x_n) = f(x) = \lim_{n \to \infty} h(x_n) = h(x)$$
                by given assumption $f(x_n) \leq g(x_n) \leq h(x_n)$ since $x_n \in D$ for all $n$. Because $f(x_n)$ and $h(x_n)$ converge to the same value, call it $s$, by the squeeze theorem we can say $g(x_n)$ converges to that same value s. Furthermore 

                    $$f(x) \leq g(x) \leq h(x)$$ 
                    $$s \leq g(x) \leq s$$

                Thus $s = g(x)$. Hence we have that $(x_n)$ converges to $x$ and $(g(x_n))$ converges to $g(x)$, thus $g$ is continious at $x$. 

            \end{solution}
        \item Exhibit a continious function $f: (0, 1] \to \R$ so that $f((0, 1])$ is not bounded. Prove all your claims
            \begin{solution}
                I claim $f(x) = \frac{1}{x}$ works. Let $(x_n)$ be a sequence in $(0, 1]$ that converges to some real number $x$. Now consider $$\lim_{n \to \infty} f(x_n) = \lim_{n \to \infty} \frac{1}{x_n} = \frac{1}{x}$$
                which we proved in the HW (note that $x_n \in (0, 1]$ so all values of $x_n$ are non-zero). So $(x_n)$ converges to $x$ and $f(x_n)$ converges to $f(x)$, thus $f$ is continious over the interval $(0, 1]$. To prove f((0, 1]) is unbounded we want to show for any $M \in \R$ there is some $x \in (0, 1]$ such that $\frac{1}{x} > M$. Assume by contradiction $f((0, 1])$ is bounded, thus there exists some $M_0 \in \R$ such that  $\dfrac{1}{x} \leq M_0$ for all $x \in (0, 1]$. Note that $M_0$ must be positive as $\dfrac{1}{x} > 0$ for all $x \in (0, 1]$. Choose $x_0 = \dfrac{1}{M_0 + 1} \in (0, 1]$, then we get $\dfrac{1}{x_0} = \dfrac{1}{\frac{1}{M_0 + 1}} = M_0 + 1 < M_0$, where the last inequality is by assumption. We have reached a contradiction thus, $f((0, 1])$ is unbounded. 
            \end{solution}
        \item Prove that if $f: D \to \R$ is continous and $C \subset D$ is compact, then $f(C)$ is bounded.
            \begin{solution}
                Suppose $f(C)$ is not bounded then there exists $x_n \in C$ such that $f(x_n) > n$ for all $n \in \N$. Let $M \in \R$ then there exists some $N \in \N$ such that $N > M$ (by archimedean principle), then for all $n > N$ we have $f(x_n) > n > N > M$. Thus $\lim_{n \to \infty} f(x_n) = \infty$. $x_n \in C$ for all $n \in \N$, since $C$ is compact we know there exists some convergent subsequence $(x_{n_k})$ that converges to some $x \in C$. By continuity of $f$, $\lim_{n \to \infty} f(x_{n_k}) = f(x)$. However this contradicts that $\lim_{n \to \infty} f(x_n) = \infty$, as we proved in class that if a sequence diveges then all of its subsequences must also diverge. Thus $f(C)$ must be bounded.
            \end{solution}
    \end{enumerate}

\end{document}
