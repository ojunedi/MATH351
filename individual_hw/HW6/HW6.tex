\documentclass[11pt]{exam}
\usepackage[margin=1in]{geometry}
\usepackage{amsfonts, amsmath, amssymb, amsthm}
\usepackage{mathtools}
\usepackage{enumerate}
\usepackage{listings}
\usepackage{array}

\usepackage{fullpage} % margin formatting
\usepackage{enumitem} % configure enumerate and itemize
\usepackage{amsmath, amsfonts, amssymb} % math symbols
\usepackage{xcolor, colortbl} % colors, including in tables
\usepackage{makecell} % thicker \Xhline in table
\usepackage{graphicx} % images, resizing
\usepackage[T1]{fontenc}
% sometimes needed packages
% \usepackage{hyperref} % hyperlinks
% \usepackage{logicproof} % natural deduction
\usepackage{tikz} % drawing graphs
\usepackage{algpseudocode} % pseudocode

% paragraph formatting
\setlength{\parskip}{6pt}
\setlength{\parindent}{0cm}

% newline after Solution:
\renewcommand{\solutiontitle}{\noindent\textbf{Solution:}\par\noindent}

% less space before itemize/enumerate
\setlist{topsep=0pt}

% creates \filcl to grey out cells for groupwork grading
\newcommand{\filcl}{\cellcolor{gray!25}}


% creates \probnum to get the problem number
\newcounter{probnumcount}
\setcounter{probnumcount}{1}
\newcommand{\probnum}{\arabic{probnumcount}. \addtocounter{probnumcount}{1}}

% use roman numerals by default
% \setlist[enumerate]{label={(\roman*)}}

% creates custom list environments for grading guidelines, question parts
\newlist{guidelines}{itemize}{1}
\setlist[guidelines]{label={}, left=0pt .. \parindent, nosep}
\newlist{gwguidelines}{enumerate}{1}
\setlist[gwguidelines]{label={(\roman*)}, nosep}
\newlist{qparts}{enumerate}{2}
\setlist[qparts]{label={(\alph*)}}
\newlist{qsubparts}{enumerate}{2}
\setlist[qsubparts]{label={(\roman*)}}
\newlist{stmts}{enumerate}{1}
\setlist[stmts]{label={(\roman*)}, nosep}



% in order to compile this file you need to get 'header.tex' from
% Canvas and change the line below to the appropriate file path
\input{"/Users/omerjunedi/Documents/math351/individual_hw/header.tex"}

% TODO: always change
\newcommand{\hwnum}{2/14}
\newcommand{\duedate}{Feb 5}
\usepackage{xcolor}

\hwheader  % header for homework
% \hwslnheader   % header for homework solutions

%Comment out this line to hid "Solution: ..." boxes.
\printanswers

\begin{document}

    \begin{enumerate}
        \item Prove that every sequence has a monotone subsequence
        \begin{solution}
            Let $(x_n)$ be a sequence \\
            \textbf{\textit{Case 1: Infinitely many dominant terms}} \\
            Let $(x_{n_k})$ be a subsequence of all dominant terms, which we can construct because 
            there are infinitely many of them. Pick some arbitrary point $n_k$ in the subsequence
            then we know that $x_{n_{k+1}} < x_{n_k}$ since $x_{n_k}$ is a dominant term. Thus for all 
            $x_{n_{k+1}} < x_{n_k}$ for all $n_k$ and $(x_{n_k})$ is monotone decreasing by definition. \\
            \textbf{\textit{Case 2: Finitely many dominant terms}} \\
            Since there are finitely many terms, there is a last dominant term. Let $x_N$ be the last dominant term, 
            then let $x_{N+1} = x_{n_1}$ the first terms of the subsequence. We know that $x_{N+1}$ is not a dominant 
            term since $x_N$ was the last dominant term. Thus there exists some $n_2 > n_1$ such that $a_{n_2} > a_{n_1}$.
            Likewise there exists some $n_3 > n_2$ such that $a_{n_3} > a_{n_2}$. Continuing this process we can construct
            an increasing subsequence, hence $x_{n_k}$ is a monotone increasing subsequence. \\
            In either case there is a monotone subsequence, thus all sequences have a monotone subsequence.
        \end{solution}    
        \item Let $k \in \R$             
            \begin{enumerate}
                \item Prove that the function $f(x) = kx$ is continious at $x=5$
                \item Prove that the function $f(x) = kx$ is continious
            \end{enumerate}
        \begin{solution}
            \begin{enumerate}
                \item Let $(x_n)$ be a convergent sequence such that $\{x_n | n \in \N\} \subseteq \R$ and that 
                converges to $5$. Then 
                $$\lim_{n \to \infty} f(x_n) = \lim_{n \to \infty} k \cdot x_n = k \lim_{n \to \infty} x_n = 5k = f(5)$$
                Thus by definition $f(x) = kx$ is continious at $x = 5$
                \item Let $(x_n)$ be a convergent sequence such that $\{x_n | n \in \N\} \subseteq \R$ and that 
                converges to $x$. Then
                $$\lim_{n \to \infty} f(x_n) = \lim_{n \to \infty} k \cdot x_n = k \lim_{n \to \infty} x_n = kx = f(x)$$
                Thus by definition $f(x) = kx$ is continious.
            \end{enumerate}
        \end{solution}
        
    \end{enumerate}

\end{document}